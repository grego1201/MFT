\section{Objetivos alcanzados}

El objetivo por el que se llevó a cabo este proyecto fue desarrollar un sistema el cual
sea capaz de ayudar a aquellas personas que se iniciaran en mundo de esgrima.
Cuando empiezas es difícil comprender que está pasando y en tus primeras competiciones
sin nadie que te guíe en cada asalto, hace que la curva de aprendizaje e introducción
al deporte sea demasiada inclinada. Desde el inicio se descartó desarrollar el sistema
completo puesto que este sería muy complejo y llevaría mas tiempo del disponible para
realizar este proyecto, por eso se decidió desarrollar un prototipo del sistema el cual
cubra el arma mas utilizada. A continuación se muestra el resumen de los objetivos planteados
desde el inicio y cuanto han sido satisfechos:


\begin{itemize}
  \item \textbf{Adquisición conocimiento:} El desarrollo de esta etapa no fue muy complicada
    puesto que se partía con una base teórica lo cuál facilitó el cumplimiento de este objetivo.
    También se contaba con una buena relación y disposición del experto, siendo este un factor
    fundamental para el éxito de este objetivo, facilitando este una gran disposición y apoyo
    en el proyecto desde el principio del proyecto. A destacar que no se captó todo el conocimiento
    necesario para desarrollar el árbol táctico completo de espada.
  \item \textbf{Creación árbol táctico:} La creación del árbol táctico fue todo un éxito gracias
    a la gran disposición por parte del experto, consolidando el conocimiento adquirido anteriormente.
    Además aportó ideas que no fueron recogidas con anterioridad. Al final quedó muy claro y completo
    de acuerdo al marco con el que se trabajaba, siendo conscientes del tiempo disponible. En definitiva,
    se abordó todo el conocimiento adquirido durante el anterior objetivo y es por esto que se
    puede decir que fue todo un éxito.
  \item \textbf{Implantación sistema de apoyo a la toma de decisiones:} El desarrollo de este
    objetivo fue quizás el mas sencillo de todos por el perfil del desarrollador del proyecto.
    Desde el inicio se tenía claro que era un prototipo, por tanto no requería de grandes diseños
    y los bocetos fueron de gran importancia para tener una idea inicial de que se quería hacer.
    Esto facilitó el desarrollo de la implantación del sistema. Por otro lado la buena definición
    de requisitos también ayudo a tener una idea global de que se quería hacer, asimismo como
    una idea de que se quería hacer en cada paso. Por otro lado, el prototipo es bastante mejorable
    en algunos aspectos que se mencionan mas adelante y es por esto que no podemos decir que se
    haya cumplido por completo con este objetivo.
  \item \textbf{Accesible desde cualquier parte:} Este objetivo fue cumplido en su mayoría, puesto
    que se hizo un despliegue con Heroku para que sea accesible en cualquier sitio con Internet,
    el cual será la mayoría de los casos, pero se han dejado algunos casos como cuando no dispongamos
    de una conexión a Internet. También el uso desde otro dispositivo que no sea un ordenador de
    escritorio.
  \item \textbf{Ampliación con IA:} Este fue, con diferencia, el apartado mas costoso del proyecto.
    Desde la búsqueda de una \acs{BBDD} que al final se tuvo que generar scrapeando los datos. Primero
    por rehacer el código varias veces debido a la remodelación de la página durante el desarrollo
    del mismo. Después la poca información disponible hizo que los datos no fueran suficientes para
    poder obtener conocimiento de estos, por lo que hubo que pasar de nuevo por el proceso de
    búsqueda y obtención de mas datos para que el conjunto de todos fueran relevantes. Durante este
    proceso hubo una nueva remodelación, por lo que prácticamente hubo que empezar de cero. Después
    hubo que exprimir al máximo para poder obtener unos resultados válidos y aún así no fueron perfectos.
    Se puede estar satisfecho con el resultado obtenido y la base que se partía, pero estos están lejos
    de ser perfectamente válidos. Es por todo esto que se puede decir que se ha cumplido parcialmente
    este objetivo.
\end{itemize}

Con todo lo mencionado anteriormente y viendo los resultados del proyecto se podría decir que ha sido
un éxito, pero no por completo. Se han alcanzado todos los objetivos propuestos, pero el resultado
de estos han sido mejorables.


