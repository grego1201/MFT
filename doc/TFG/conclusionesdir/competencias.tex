\section{Competencias}
En la \hyperref[tab:competencias adquiridas]{tabla 6.1} se listan las competencias alcanzadas específicas de la intensificación
(computación) junto a su justificación.

\begin{longtable}{p{.45\textwidth} p{.45\textwidth}}
  \caption{Competencias adquiridas}
  \label{tab:competencias adquiridas}

  \endfirsthead
  \endhead

  \textbf{Competencia} & \textbf{Justificación} \\
  \hline
  \rowcolor[HTML]{D6D6D6}
  Capacidad para conocer y desarrollar técnicas de aprendizaje computacional y diseñar e implementar aplicaciones y sistemas que las utilicen, incluyendo las dedicadas a extracción automática de información y conocimiento a partir de grandes volúmenes de datos. & Durante el desarrollo de este TFG se ha generado una base de datos de la cual se ha extraído un modelo de aprendizaje automático con el que poder reforzar el entrenador.\\

  Capacidad para desarrollar y evaluar sistemas interactivos y de presentación de información compleja y su aplicación a la resolución de problemas de diseño de interacción persona computadora. & Se ha generado un aplicación mediante la cual un usuario puede interaccionar con ella para obtener respuestas a sus preguntas.\\

  \rowcolor[HTML]{D6D6D6}
  Capacidad para adquirir, obtener, formalizar y representar el conocimiento humano en una forma computable para la resolución de problemas mediante un sistema informático en cualquier ámbito de aplicación, particularmente los relacionados con aspectos de computación, percepción y actuación en ambientes entornos inteligentes. &  Se ha formalizado conocimiento técnico y experiencia de un experto en la materia de esgrima y se ha plasmado en un programa como consecuente hemos obtenido un conjunto de reglas de las cuales mediante unas entradas se puede obtener una salida.  \\

  Capacidad para evaluar la complejidad computacional de un problema, conocer estrategias algorítmicas que puedan conducir a su resolución y recomendar, desarrollar e implementar aquella que garantice el mejor rendimiento de acuerdo con los requisitos establecidos. & Se han estudiado las complejidades de los algoritmos utilizados de modo que se halló la solución mas óptima para cada caso. \\

  \hline
\end{longtable}




