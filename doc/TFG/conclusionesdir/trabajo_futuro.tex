\section{Trabajo futuro}

Como se indicó al comienzo de este documento este sistema no es más
que un prototipo. Es por esto que es muy susceptible a mejoras
debido a que no está completado. En este apartado hablaremos de las
debilidades que presenta actualmente el sistema. Después hablaremos
sobre las principales mejoras que se podrían hacer a este para subsanar
dichas flaquezas, además de otras para ampliar y mejorar el sistema.
Este proyecto estará en continua evolución debido a que el conocimiento
y las tácticas variarán con los años como se ha podido ver en la historia
desarrollando diferentes estilos en esgrima. También se crearán nuevos
equipos como los puños, lo cual hará que haya que adaptarse a ellos. Asímismo
las reglas van cambiando como este año cambió la pasividad, lo cual hace
que haya que cambiar ciertas partes de la estrategia, permitiendo hacer
cosas nuevas o evitando que se hagan otras.

\subsection{Carencias}
\begin{enumerate}
  \item \textbf{Limitación armas:} La primera carencia que nos encontramos es
    la limitación en cuanto a las armas de este deporte. Solamente contamos
    con una de las tres existentes. A pesar de hacer el sistema para la mas
    popular, no se ofrece ayuda a todo el público posible, por lo que se deberá
    trabajar en ampliar para poder cubrir las tres armas.
  \item \textbf{Tiempos carga:} A pesar de desarrollar dos modos para facilitar
    y agilizar los tiempos para cuando necesitemos una respuesta mas rápida (recordemos
    que los descansos entre asaltos son de un minuto) para la versión completa
    necesitamos, en el mejor de los casos, mínimo entre tres y cuatro segundos
    para que se calcule el resultado. Esto es debido a que la llamada la API
    tarda ese tiempo en calcular la predicción. Esto puede ser un problema para
    usuarios con poca paciencia. Otro de los problemas relacionados con los tiempos
    es el primer acceso a la aplicación. El plan gratuito de Heroku apaga los servidores
    al no recibir peticiones en treinta minutos, lo cual hace que se tengan que levantar
    de nuevo en el primer acceso, tardando estos unos 25 segundos.
  \item \textbf{Experiencia usuario web:} La aplicación web ha sido desarrollada
    en poco tiempo debido al límite de horas dispuestas para este \acs{TFG}. Otro factor
    a tener en cuenta es que el desarrollador de la aplicación no es experto
    en desarrollar interfaces. Por los motivos expuestos anteriormente el usuario
    final puede no tener la mejor experiencia y actualmente es bastante mejorable,
    desde los formularios haciendo que estos sean mas intuitivos, rápidos y explicativos
    por si mismos. No solo los formularios deberán ser repasados y mejorados, también la forma
    en la que se dan los resultados, los colores, la distribución del menú, etc.
  \item \textbf{Limitación web:} Solo se ha desarrollado una aplicación web y no fue
    adaptada para móvil por falta de tiempo. Uno de los usos de la aplicación es
    en competición, por lo que no podrás estar con un ordenador portátil con conexión
    a internet. Es por esto que no se verá del todo bien desde el navegador del móvil
    y no se cubren los casos en los que no tengamos Internet, como cuando viajemos
    al extranjero y no contemos con una tarifa para poder navegar por Internet.
  \item \textbf{Predicción:} El sistema de predicción terminó con un porcentaje
    aproximado de 65\% de acierto, es por esto que se deberá tener en cuenta
    debido a que es mejor que tirar una moneda al aire, pero su fiabilidad
    no esta cerca de ser totalmente fiable.
  \item \textbf{Internacionalización:} La aplicación actualmente solo está
    disponible en dos idiomas, cierto es que de los mas comunes y utilizados
    no se llega a todos los tiradores de todos los países. Faltan idiomas
    en los que esgrima esta muy presente como húngaro, francés e italiano.
\end{enumerate}

A continuación se hablará sobre las mejoras que se podrán hacer en cada apartado
del sistema, dando una primera aproximación sobre la solución a tener en cuenta y,
en caso de que corresponda, que problema subsana.

\subsection{Sistema experto}

\begin{itemize}
  \item \textbf{Lógica difusa:} Añadiendo lógica difusa al sistema experto hará que las reglas,
    junto a los resultados, sean bastante mas precisos. Esto hará que el usuario final
    tenga mas claro como de certera es una acción.

  \item \textbf{Ampliar reglas:} Se han dejado numerosos movimientos, técnicas y tácticas
    fuera del sistema experto, es por esto que habrá que seguir trabajando con el
    experto, buscando nuevos a ser posible para tener diferentes puntos de vista,
    para cubrir todo el árbol táctico de espada. Para ello se deberá seguir
    trabajando igual que se hizo anteriormente, con sistema de entrevistas, analizando
    los resultados anteriores y confirmando el conocimiento adquirido con el experto.
    Después se traspasará dicho conocimiento al sistema experto. Otro punto a tener
    en cuenta es que actualmente solo se está presentando el caso inicial de un
    asalto. Un horizonte nuevo a explorar en el desarrollo del sistema sería
    introducir las acciones y/o tácticas ya realizadas con el resultado que obtuvieron
    y que el sistema sea capaz de re-calcular los resultados en función de estos
    datos introducidos.
\end{itemize}

\subsection{API predicción}
\begin{itemize}
  \item \textbf{Mejorar fiabilidad predicción:} El sistema de predicción no obtuvo un
    porcentaje de fiabilidad muy alto, es por esto por lo que se deberá trabajar
    en mejorarla. Para ello el primer paso será ampliar la \acs{BBDD}. Lo primero y mas sencillo
    será obtener mas registros. Esto es fácil de realizar puesto que la base de obtención
    de datos ya está hecha, solo habría que renovar el código puesto que es posible
    que la página haya cambiado y ya no sirva, pero se juega con la carta de que
    hay varias competiciones al mes entre las diferentes categorías. Es por esto
    que será fácil ampliarla. Otro objetivo a conseguir será ampliar las columnas
    de cada registro. Se puede intentar conseguir mas características de los tiradores
    como la altura de estos, categoría (no es lo mismo que se enfrenten dos de la misma,
    que un junior y un senior, aunque esto se subsana en parte con la edad), ranking
    obtenido en poules puesto que esto marca como estaba el tirador en ese día, etc.
    Con todo esto podremos mejorar el modelo de entrenamiento y con esto aumentaremos
    la fiabilidad del sistema.

  \item \textbf{Mejorar tiempos API calcular predicción:} Actualmente la API tarda entre
    tres y cuatro segundos (suponiendo que la máquina esté levantada y a pleno funcionamiento)
    lo cual hace que no podamos incluir esto en toda la parte de la aplicación. El método
    que corresponde con la llamada en la API junta el entrenamiento del modelo junto
    al cálculo de predicción en la llamada. Lo que mas tarda con diferencia, llegando a ser
    del 95\% del tiempo es el entrenamiento del modelo. Es por esto que si se separara
    el entrenamiento del modelo, que con hacerse una vez debería ser suficiente, del
    cálculo de predicción se debería ahorrar mucho tiempo en la llamada de predicción.
    El resultado de este cambio debería ser dejar la respuesta de dicho endpoint prácticamente
    inmediata
\end{itemize}

\subsection{Aplicación web}
\begin{itemize}
  \item \textbf{Mejorar UX web:} Para reducir los tiempos que tarda el usuario y mejorar
    la comprensión de estos se deberá rehacer los formularios o mejorar ciertos aspectos
    como los seleccionables siendo estos mas específicos. También se deberá añadir una
    ayuda rápida y explicativa de que es cada campo y para que sirve. Por otro lado
    habrá que revisar la distribución de los botones para ahorrar tiempo y que sean mas
    fáciles de encontrar cada sección.
  \item \textbf{Mejorar mensajes feedback resultados:} Los resultados son poco explicativos
    debido a que solo te indican si debes realizar una acción o no. Esto es útil para aquellos
    usuarios que tengan cierta experiencia y puedan discernir que es dicha acción. En un futuro
    se deberá ampliar estos mensajes, indicando que es dicha acción o táctica y porque se ha
    tomado dicha decisión.
  \item \textbf{Sección FAQ:} Prácticamente toda web con este formato tiene un apartado de
    preguntas frecuentes el cuál no ha sido desarrollado por falta de tiempo. Incluir este apartado
    puede ayudar positivamente a aquellos usuarios que tengan ciertas dudas sobre el sistema
    o incluso con el mundo de esgrima.
  \item \textbf{Foro:} Incluir un foro puede ser muy enriquecedor para la aplicación debido a que
    los usuarios pueden compartir diferentes opiniones sobre los resultados y ayudarse entre ellos.
    Esto abre la posibilidad a generar una comunidad alrededor de la aplicación, haciendo de esta
    un factor mas a tener en cuenta en el mundo de esgrima.
  \item \textbf{Usuarios:} Añadiendo usuarios se podrá habilitar y mejorar la posibilidad del foro
    mencionada anteriormente. También abre otras tantas puertas como la gestión de perfiles de usuario
    guardando el idioma por defecto, sin tener que cambiarlo cada vez que acceda a la aplicación.
    También se podrán guardar perfiles de tiradores, pudiendo hacer las consultas al sistema
    cargando dichos tiradores en vez introducir sus características, agilizando de este modo
    el proceso. Asimismo se podrá guardar el historial de las consultas realizadas, ahorrando
    consultas, evitando cargas extra al sistema y ahorrando tiempo al usuario.
  \item \textbf{Adaptación responsive:} La aplicación solo fue desarrollada pensando en el
    uso desde el ordenador de escritorio. Es por esto que se deberá seguir trabajando en la
    adaptación a diferentes dispositivos como los móviles, tablets, etc. También se deberá
    facilitar el uso para personas con cierta discapacidad visual, adaptando la web a dicha
    para que el uso de esta para dichas personas sea lo mas cómodo posible.
  \item \textbf{Mejora SEO:} Dedicando recursos al SEO conseguiremos que la aplicación web
    sea mas fácil de encontrar para los usuarios, haciendo que sea mas conocida la página y
    por tanto mas utilizada.
\end{itemize}

\subsection{General}
\begin{itemize}
  \item \textbf{Sistema experto para entrenamiento:} Durante el proceso de desarrollo de
    la aplicación y tras recibir feedback de los usuarios que la utilizaron en las pruebas
    se llego a la conclusión de que un apartado de entrenamiento aportaría mucho a la aplicación.
    Es obvio que esto es algo totalmente diferente al sistema de apoyo a la toma de decisión,
    pero si se complementan debido a que esta parte puede ayudar mucho a los clubes pequeños
    a perfeccionar y ampliar sus entrenamientos. Se hizo un primer desarrollo indicando
    acciones de manera aleatoria para trabajar durante los asaltos de entrenamiento. Pero el
    objetivo principal de esto es desarrollar un sistema experto el cuál nos indique un
    entrenamiento a realizar en función de nuestras fortalezas y debilidades, nuestras
    características como tiradores (altura, puño utilizado, estilo de esgrima,
    acciones conocidas, etc), tiempo disponible y forma física de la que se parte.
  \item \textbf{Cambiar a plan pago para mejorar tiempos:} Una de las limitaciones
    impuestas por el plan gratuito de Heroku son los tiempos de inicio de los servidores
    siendo estos de unos 25-30 segundos. Ampliando el plan a uno de pago, siendo este
    de 7 dolares al mes, evitaremos que se apaguen los servidores. También podremos
    comprar un dominio par que el acceso a este dando una mejor imagen de la página.
    Para esto se podrá buscar algún tipo de financiación como la publicidad dentro
    de la aplicación o algún tipo de subvención de las diferentes federaciones, tanto
    regionales como la española, incluso llegando a presentar a la internacional el
    proyecto.
  \item \textbf{Añadir sección florete y sable:} Como se menciono antes el sistema
    solo está pensado para espada, faltando florete y sable. Añadiendo estas dos
    armas conseguiremos cubrir todo el público. Para ello deberemos seguir
    todo el proceso llevado a cabo con la espada, desarrollando un sistema experto
    basado en conocimiento, extrayendo este de un experto en la materia.
  \item \textbf{Crear aplicación móvil offline:} Uno de los problemas
    que se mencionó antes fue la falta de acceso a la aplicación cuando no se
    tiene acceso a Internet, además de la dificultad de uso desde el móvil.
    Por estas razones se deberá desarrollar una aplicación móvil, tanto android
    como IOs. Del mismo modo también será conveniente desarrollar una aplicación
    de escritorio para aquellos usuarios que utilicen un PC en sus lugares de entrenamiento
\end{itemize}
