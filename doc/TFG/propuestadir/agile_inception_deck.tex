\section{Agile Inception Deck}

Antes del inicio del desarrollo será necesario ponerse de acuerdo con todos los miembros involucrados en el proyecto.
Para ello se utilizará \textit{Agile Inception Deck} el cual estará formado de un conjunto de dinámicas las cuales permitirán
ayudar a llevar al proyecto a la dirección deseada. La parte de \textit{Inception} estará formada por diez preguntas, las
cuales se detallan a continuación:

\begin{enumerate}
  \item \textbf{¿Por qué estamos aquí?}

    Esta cuestión nos permitirá contextualizar el proyecto, respondiendo el motivo del mismo.
  \item \textbf{El \textit{Elevator pitch}}

    Literalmente sería el discurso del ascensor. Este apartado intentará dar respuesta a las preguntas qué, por qué y para qué.
    Para ello se utilizará el símil del tiempo que dura un viaje en el ascensor.
  \item \textbf{Diseñar una caja para el producto}

    Se dará una visión del producto desde la perspectiva del cliente.
  \item \textbf{Lista de los no}

    Aquí detallaremos que no es el producto y que no está dentro de él.
  \item \textbf{Conoce a tus vecinos}

    Daremos a conocer a las personas que están involucradas en el proyecto.
  \item \textbf{Muestra la solución}

    Se mostrará la metodología y arquitectura que se utilizará en el desarrollo del proyecto.
  \item \textbf{¿Qué nos quita el sueño por las noches?}

    En este apartado se intentarán predecir los posibles imprevistos surgidos durante
    el desarrollo del proyecto.
  \item \textbf{Tamaño del proyecto}

    Planificación a alto nivel de la duración del proyecto.
  \item \textbf{Muestra con claridad lo que se va a dar}

    Identificar los temas más importantes en el desarrollo del proyecto
  \item \textbf{Muestra lo que va a conllevar}

    Análisis de costes del proyecto.
\end{enumerate}
