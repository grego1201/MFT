\chapter{Evaluación y Resultados}
\label{cap:Evaluación y resultados}

En este capítulo se hablará sobre la evaluación del sistema y los resultados que se obtuvieron.
Para ello se hicieron diferentes pruebas. Recordar que los casos han sido validados mediante
test y estudiados con el experto que nos ayudó a obtener conocimiento sobre el tema.
La primera de ellas fue una evaluación con diferentes
expertos en la materia. Dichos expertos son tanto maestros de salas de armas como tiradores
de alta competición a nivel nacional. No todos dieron permiso para que su nombre saliera, por
lo que se decidió mantener el anonimato de todos ellos para evitar posibles descartes y relaciones.
A cada uno de ellos se les dio acceso a la herramienta para que pudieran probarla y dar feedback
después sobre la misma.

A continuación se puede ver un resumen de las evaluaciones de los expertos consultados:
\begin{itemize}
  \item \textbf{Experto 1:} % Buena idea pero se queda corta de momento
  \item \textbf{Experto 2:} % Le hubiera venido bien cuando empezó
  \item \textbf{Experto 3:} % No cree que sea útil puesto que esto es algo que te da la experiencia
  \item \textbf{Experto 4:} % Puede ser útil pero no para empezar puesto que no se tiene conocimiento suficiente
  \item \textbf{Experto 5:} % Buena herramienta de entrenamiento con gran futuro si se sigue desarrollando, se la dará a sus alumnos
  \item \textbf{Experto 6:} % Mejorable pero útil desde ya para aquellos que empiezan
\end{itemize}

Otra prueba de satisfacción que se llevo a cabo fue una encuesta realizada mediante un formulario
anónimo. Las preguntas formuladas se hicieron con pretensión de conocer cómo de útil les resulto
la herramienta, si la recomendarían y un campo libre para que puedan explicar mejor su opinión.
También se pregunto acerca de los años que se llevaba practicando esgrima, para sacar mas información
acerca del posible conocimiento y público que contestaba a esas preguntas.
Los resultados obtenidos se pueden ver en la figura X.x

% Falta por sacar imagen resultados

También se utilizó en un entorno cercano. Durante la competición del día 30 de Noviembre se llevó
a cabo el primer ranking de la temporada 2019-2020 de menores de la federación de Castilla-La Mancha.
En dicha competición se facilitó el acceso a la herramienta a Maria Rosa Maestre, tiradora del club
Espadas de Calatrava. En su fase de grupos tenía un varios enfrentamientos complicados que en anteriores
ocasiones le costo resolver, obteniendo una derrota como resultado en la mayoría de ellos. En el día
del torneo usando la herramienta consiguió sorprender a las tiradoras rivales con nuevas tácticas que
antes no había llevado a cabo. Al hablar con ella se veía mas confianza por lo que este también
pudo ser un factor que le llevara a la victoria. El resultado final para la tiradora fue de
un gran tercer puesto, perdiendo en semifinales en un asalto duro físicamente cuyo ritmo no pudo
seguir. Eligió bien las acciones pero la diferencia vino marcado por el físico de ambas tiradoras.
A destacar que fue quien acabó ganando la competición, esto indica que fue una gran rival contra
quien cayó.

Además de se llevó a cabo un experimento en la sala de entrenamiento del club de esgrima Espadas
de Calatrava. Se estudiaron los resultados obtenidos entre diferentes tiradores habituales de
la sala, anotando el número de victorias y derrotas acumulados entre cada uno de ellos junto
a los puntos anotados en cada uno de ellos. El experimento de evaluación consistía en no avisar
al resto de compañeros de cuando el sistema estaba siendo utilizado para evitar la mayor sugestión
posible. Las características de cada tirador se pueden ver en la tabla X.x.

Los resultados obtenidos antes de utilizar el sistema fueron los siguientes:

% Hacer tabla resultados pre-sistema

Mientras que los resultados obtenidos después de utilizar el sistema en los tiradores 1 y 3 fueron
los siguientes:

% Hacer tabla post-sistema

% Sacar conclusiones de mejoras, blablaba
