\chapter{Resumen}

Llegar a dominar cualquier deporte es muy complicado y por lo general se necesita
 de alguien que te guíe. Esta persona hará que aprendas mas rápido ya que
 ahorrará todo el tiempo que le tomó adquirir el conocimiento, transmitiendo este
 mediante enseñanzas o ejercicios. Muchos deportes se basan en la toma de
 decisiones para después ejecutarlas y es la mezcla de ambas quien decide el
 ganador de un enfrentamiento. Esta decisión puede ser tomada por el deportista o
 entrenador si se dispone del mismo.

La entrada al mundo competitivo en un deporte suele ser compleja y más si son de
 enfrentamientos directos donde la táctica elegida por cada uno, adaptando esta a
 lo largo del mismo, determinará, entre otras cosas, quien
 será el ganador.

En este documento se narra el proceso llevado a cabo para desarrollar un prototipo
 funcional de un sistema de apoyo a la toma de decisión. En concreto para
 esgrima. La funcionalidad de este ha sido acotada al tiempo estimado de duración
 del proyecto pero se podrá, y deberá, seguir trabajando sobre este para completar
 su desarrollo.

Lo primero que se hizo fue un estudio de viabilidad para prever el éxito del
 proyecto. El desarrollo comenzó tomando como base un sistema experto basado en
 el conocimiento extraído mediante diferentes técnicas de Ingeniería del Conocimiento, como por ejemplo entrevistas. Acto seguido se quiso ir un paso
 mas lejos y se utilizaron técnicas de Inteligencia Artificial obteniendo
 los datos mediante técnicas de WebScrapping y se extrajo conocimiento de estos
 mediante un proceso KDD.

Todo lo anterior fue conectado mediante una interfaz de usuario vía web para
 poder tener acceso al sistema desde cualquier dispositivo con Internet. Además,
 se desarrolló una pequeña API para una parte del sistema de apoyo a la toma de decisión.


\chapter{Abstract}

Mastering any sport is very complicated and usually requires
of someone to guide you. This person will make you learn faster because
will save all the time it took to acquire the knowledge, transmitting this
through teaching or exercise. Many sports are based on taking
decisions and then execute them, and it is the mixture of both that decides the
winner of a match. This decision can be made by the athlete or
coach if you have one.

The entrance to the competitive world in a sport is usually complex and more if they are
direct confrontations where the tactic chosen by each one, adapting this to
the length of it, will determine, among other things, who
will be the winner.

This document describes the process carried out to develop a functional prototype
of a decision support system. Specifically for
fencing. The functionality of this has been limited to the estimated time
of the project, but further work can and should be done on it to complete
its development.

The first thing that was done was a viability study to foresee the success of the
project. The development started on the basis of an expert system based on
knowledge extracted through different Knowledge Engineering techniques, such as interviews. Then we wanted to go one step further
and used Artificial Intelligence techniques obtaining
the data by means of WebScrapping techniques and knowledge was extracted from these
through a KDD process.

All of the above was connected via a web-based user interface to
to be able to access the system from any device with Internet. In addition,
a small API was developed for one part of the decision support system.