\newline

\section{\acs{DSS} usando técnicas de aprendizaje automático en esgrima}
\subsection{\acl{DSS}}

Para cualquier persona, en cualquier ámbito siempre se ha querido tomar una buena decisión y no siempre se ha podido, en la mayoría de los casos
 por no tener los conocimientos suficientes. Pero ¿qué es la mejor, una buena y una mala decisión?
 Esto es algo que, dependiendo del contexto, podría ser tanto subjetivo como objetivo. Por ejemplo, si nuestro
 objetivo es conseguir correr una maratón, no entrenar para ello posiblemente sea una mala decisión.
 Sin embargo, si hemos decidido seguir un plan de entrenamiento, el cual nos preparará para terminar
 nuestro objetivo habremos tomado una buena decisión. Pero esto sigue sin habernos contestado la pregunta
 de cuál será la mejor decisión. Bien, en este caso tendremos que entrar mas en detalle para saber
 cual es nuestro objetivo específico, en el caso de que solo sea terminarla, la mejor decisión será seguir aquel
 plan de entrenamiento que con el menor esfuerzo nos permita terminarla. Sin embargo, si nuestro objetivo
 es el de conseguir una buena marca, esta no habrá sido la mejor decisión puesto que tendremos que
 seguir un plan el cual nos permita ir mas rápido, aunque el esfuerzo sea mayor.

Una vez aclarada que es una mala, buena y la mejor decisión podremos hablar sobre lo que es un sistema
 de apoyo a la toma de decisiones. Estos son sistemas desarrollados para dar una decisión, en su mayoría
 aplicaciones informáticas, los cuales siguen el proceso de una toma de decisiones. Estos utilizan
 los datos y modelos para generar las alternativas posibles y acabar tomando la mejor decisión. Detrás
 de estos sistemas suele haber un sistema experto detrás, el cuál es un conjunto de reglas obtenidas
 a través de conocimiento extraído previamente. Dicho conocimiento se puede apoyar de otros sistemas como
 arboles de decisión o redes bayesianas para obtener conocimiento.

Un ejemplo de lo mencionado anteriormente fue en el \acs{DSS} basado en redes bayesianas con una aplicación
 a la lucha contra las infecciones nosocomiales \cite{hela}. En este caso lo que se pretendía
 era reducir el número de pacientes infectados. Para ello desarrollaron un sistema para ayudar a los
 médicos a prevenir dichas infecciones. Dicho sistema fue desarrollado mediante un proceso \acf{KDD} el
 cuál se nutre de una base de datos cedida por un hospital. Dicho proceso consistía en analizar los datos
 e intentar obtener conocimiento de ellos. Para ello habría que seguir una serie de pasos como la
 limpieza, pre-procesamiento de datos, detección de patrones y ahí es cuando se podría obtener conocimiento.
 Dicho sistema ayudaría a reducir costes en los hospitales además de reducir el número de casos de infección.

\subsection{Sistemas basados en el conocimiento}

Parte de un \acs{DSS} pueden ser los sistemas basados en conocimientos, los cuales parten de un conocimiento extraído de un
 experto y es transportado a una aplicación informática la cual será lo mas fiel a reproducir
 la decisión de dicho experto. Para ello se siguen una serie de procesos para poder transformar
 dicho conocimiento hacia la aplicación. Esta suele contar con una interfaz de usuario para que
 sea lo mas cómoda posible. El proceso no es trivial puesto que se necesita mínimo
 de un ingeniero de conocimiento y de un experto en la materia. Además ambos deben estar predispuestos
 a llevar a cabo el proyecto, puesto que es de vital importancia que se colabore en la mayor
 medida posible. Para ello se deberá extraer el conocimiento del experto mediante diversos métodos
 como puede ser un sistema de entrevistas, en las cuales el ingeniero le plantea una serie
 de problemas y/o dudas al experto y este deberá responderle. Una vez finalizada el ingeniero
 analizará el resultado de la entrevista, pasando a reglas dicho conocimiento o elaborando
 una nueva entrevista en caso de que fuera necesario.

Un ejemplo de lo mencionado anteriormente sería el sistema desarrollado en la universidad de Split \cite{sportTalent}.
 Aquí desarrollaron un sistema experto el cual era capaz de detectar talento en jóvenes que
 practicaban diferentes deportes con sus modalidades. En este caso iban un paso mas allá puesto
 que no solo jugaban con el conocimiento extraído del experto, si no que generaban una serie de
 pesos para sus reglas en conjunto de una solución ya hecha en los deportes. Esto llevó a lograr
 una aplicación en tiempo real mediante una página web en la cual cualquiera podría realizar consultas
 acerca de si una persona podría ser talentosa. A pesar de todos los esfuerzos en el propio artículo
 mencionan que no hay una respuesta definitiva y completa a la pregunta.

Por tanto nos encontramos ante la siguiente pregunta ¿Cómo nos podríamos aprovechar los
 sistemas basados en conocimiento, sistemas de apoyo a la decisión y todo lo mencionado
 anteriormente en un campeonato de esgrima? Bien pues la respuesta es desarrollar un \acs{DSS}, ayudado de un \acs{SBC}. Esto nos permitirá tener una
 visión mas en el campo de batalla, donde
 nuestras capacidades para tomar decisiones están limitadas, ya sea por cansancio físico, presión del
 momento, etc. Esto no sustituirá a un entrenador, puesto que hay cosas subjetivas y que requieren
 de un mayor contexto que se le pueda dar al sistema, además de la \textit{intuición} que se tiene en esos momentos
 pero si nos servirá para tener una visión más de lo que se puede hacer, que en la mayoría de los
 casos nos hará falta. A pesar de todo esto recordamos que no tomará la decisión por nosotros,
 sino que nos aconsejará y será el usuario final el que deba tomar la decisión.

\subsection{\acl{ML}}
\acf{ML} es una disciplina científica que se encuentra dentro de la Inteligencia
 Artificial en la cual se crean sistemas que aprenden automáticamente. Entendemos por aprender
 como la detección de patrones complejos en gran cantidad de datos. En este caso, quien aprenderá
 será un algoritmo que revisará los datos y será capaz de predecir comportamientos en un futuro.

Dentro del \acs{ML} tenemos dos tipos de aprendizajes:

\begin{itemize}
  \item \textbf{Aprendizaje supervisado:} En este tipo tendremos una serie de datos de entrenamiento
    los cuales habrán sido etiquetados previamente. Con este tipo de aprendizaje tendremos clasificadores
    automáticos sin tener que programarlos. Podremos elegir la forma que tendrán.
  \item \textbf{Aprendizaje no supervisado:} Este tipo de aprendizaje es el recurrido cuando
    los datos que tenemos no están etiquetados para el entrenamiento. Es por esta razón por la que
    se procura encontrar algún patrón que simplifique el análisis.
\end{itemize}

Usaremos diferentes técnicas de ML para obtener conocimiento de una base de datos, de este modo
podremos reforzar nuestro \acs{SBC}, haciendo de este un sistema mucho mas
completo.
