\section{Iteración 4}

Para darle una mayor accesibilidad al sistema se decidió desarrollar
una aplicación web, de este modo cualquier persona con conexión a
Internet podría acceder a ella. El siguiente paso en el desarrollo
del proyecto sería darle acceso a cualquier persona con internet.

Para ello tendrá que estar alojada en algún servidor externo.
Puesto que no guardamos ningún dato sensible, no hace falta
que esté alojado en Europa. Después de analizar las distintas posibilidades
se decide usar Heroku. Para ello se hizo un estudio de las herramientas
de hosting mas conocidas cuyo resultado se puede ver en la \hyperref[tab:Tabla comparativa precios hosting]{tabla 5.7}

\begin{longtable}{|p{0.1\linewidth}p{0.1\linewidth}p{0.1\linewidth}p{0.1\linewidth}p{0.15\linewidth}p{0.15\linewidth}p{0.15\linewidth}|}
  \caption{Tabla comparativa precios hosting}
  \label{tab:Tabla comparativa precios hosting}
  \endfirsthead
  \endhead
  \hline
  \multicolumn{1}{|l}{Hosting} & Precio & Dominio incluido & Soporte incluido & Documentación & Retraso en primer acceso & Nº máximo de instancias \\ \hline
  Heroku & 0€ & No & No & Media & 30 segundos & 5 \\ \hline
  AWS & 0.69€ & No & Si & Alta & 0 segundos & 1 \\ \hline
  Azure & 0.49€ & No & Si & Alta & 0 segundos & 1 \\ \hline
  IBM Cloud & 0€ & No & No & Media & 30 segundos & 1 \\ \hline
\end{longtable}
