\section{Iteración 1}

Una vez establecido el tema a tratar en el proyecto y el alcance del mismo,
se procede a desarrollar cada una de las iteraciones que lo componen.

En la primera iteración se ilustra el proceso de adquisicón del conocimeinto,
lo que dará una visión mas detallada del proeycto y establecerá las bases
para el desarrollo del arbol táctico que nos ayudará en la toma de decisiones en un asalto de esgrima.

\subsection{Adquisición conocimiento básico}

Lo primero que se hizo fue una investigación previa para asentar unas bases de conocimiento
básicas sobre la materia en cuestión. Para ello se hicieron diversas búsquedas por internet
buscando información acerca del deporte y su competición. Sacando bastante información
en las páginas de la federación internacional y española. Además de diversos medios como
el podcast \textit{Llamada a pista} y redes sociales como \textit{weareeelgato}.

Sobre estas búsquedas se pudo sacar información acerca del reglamento, el cuál es algo
básico de entender para saber como se puede plantear una estrategia. También se obtuvo
información sobre los movimientos básicos de la esgrima, de este modo podríamos entender
la jerga del deporte cuando nos reunieramos con los expertos. En cuanto al contenido
multimedia se pudo sacar en claro como era un asalto de esgrima, afianzando todo el conocimiento
obtenido anteriormente, viendo los movimientos que realizaban los tiradores y como
se aplicaban las normas en los distintos casos. Además, se pudo ver como estos adaptaban
sus estilos según transcurría el combate, por lo que nos hacía ver como se planteaba
una táctica u otra en función de las características del asalto.

\subsection{Adquisición conocimiento mediante entrevistas}
%Explicar las entrevistas que se hicieron, como fueron, como se analizaron y las conclusiones
%que se hicieron sobre cada una.

Una vez que ya teníamos un conocimiento básico sobre la esgrima en la modalidad de espada,
estabamos preparados para la primera entevista con el experto. En la primera entrevista el
objetivo principal no era más que poner en claro nuestros conceptos básicos para asegurarnos
que podíamos entendernos con el experto, aunque no supieramos el razonamiento de cada caso,
pero sí el que nos decía. Es por esto por lo que se decidió una entrevista abierta semi-estructurada.

La primera entrevista se dividiría en dos partes principales. La primera de ellas sería
asegurar los conceptos adquiridos en la materia y aclarar los que no tuvieramos seguros.
Además se corregirían aquellos que fueran erroneos. La segunda parte de la entrevista
sería prácticamente abierta en su totalidad, ya que sería el experto el que nos tendría
que introducir en la materia de escoger una táctica u otra. Para ello fuimos con unas
preguntas básicas, las cuales son comunes a la mayoría de procesos de toma de decisiones.

Una vez finalizada la entrevista se analizarían los resultados y sacarían conclusiones de la misma.
El documento de la primera entrevista se puede ver en el Anexo B.

Con la primera entrevista finalizada y después de analizar los resultados sacamos en claro
las siguientes conclusiones:

\begin{itemize}
  \item Lo primero en lo que hay que fijarse para planificar una estrategia es en la distancia,
    físico y experiencia.
  \item La distancia se valorará según la diferencia de altura entre tiradores y el puño usado
    de cada uno.
  \item La experiencia puede jugar a tu favor si eres el que más tiene.
  \item El físico influye a la hora de escoger como quieres llevar el asalto.
  \item La personalidad del tirador influye en su forma de tirar. Intentar averiguar como es esa persona
    antes de un asalto.
  \item No hay una preferencia ante ceder terreno o llevar al rival a final de pista. Dependerá de la
    situación del asalto.
\end{itemize}

Además sacamos las siguientes palabras para el glosario:

\begin{enumerate}

  \item Cualidades del tirador
  \item Distancia
  \item Experiencia
  \item Físico
  \item Técnicas
  \item Tocar
  \item Entrar con hierro
  \item Entrar con segunda intención
  \item Altura
  \item Puño
  \item Francés
  \item Anatómico
  \item Más joven
  \item Más rápido
  \item Sentirse intimidado
  \item Saber mucho
  \item Ganar rápido
  \item Echado hacia delante
  \item Esperar
  \item Provocar errores
  \item Contra
  \item Ser ofensivo
  \item Alcance
  \item Poco contacto en hierro
  \item Perder distancia
  \item Puntería
  \item Ser pasivo
  \item Parar bien
  \item Buen contrataque
  \item Dejar corto
  \item Dejar pensar

\end{enumerate}

También actualizamos las tablas de objeto, atributo y valor (ver cuadro 5.3)

\begin{table}[]
  \centering
  \caption{Tabla objeto atributo y valor}
  \label{tab:Tabla objeto atributo y valor}
  \begin{tabular}{lll}
    Objeto & Atributo & Valor \\ \hline
    \multicolumn{1}{l|}{\multirow{4}{*}{Tirador}} & Puño & \{Francés, Anatómico\} \\ \cline{2-3}
    \multicolumn{1}{l|}{} & Altura & {[}0, 230{]} cm \\ \cline{2-3}
    \multicolumn{1}{l|}{} & Intimidado & \{Si, No\} \\ \cline{2-3}
    \multicolumn{1}{l|}{} & Edad & {[}0, 120{]} Años \\ \cline{2-3}
  \end{tabular}
\end{table}

Una vez reposado el conocimiento adquirido podremos pasar a la siguiente entrevista,
de este modo ampliaremos la batería de posibilidades.

El planteamiento de la segunda entrevista era afianzar los conocimientos adquiridos en
la primera y ampliar estos. Para ello lo primero que se hizo fue hacer preguntas abiertas
sobre ciertas acciones en concreto. De esta manera nos aseguraríamos que estamos entiendo
bien el contexto del problema. Después pasaríamos a ampliar el conocimiento, de nuevo
con preguntas abiertas para conseguir desarrollar un esquema táctico mas amplio y complejo.

\begin{itemize}
  \item El físico es importante pero no lo único a tener en cuenta. Los reflejos será
    algo muy a tener en cuenta junto a la velocidad de reacción.
  \item La teoría no lo es todo, también habrá que detectar como de bueno es el rival
    en las dos acciones principales de todo arte marcial, la defensa y el ataque.
    Con esto podremos trazar una táctica mas fiable y eficaz.
  \item La confianza en cada uno de los tiradores será determinante para saber
    que podemos movimientos podremos llevar a cabo o no.
\end{itemize}

Además sacamos las siguientes palabras para el glosario:

\begin{enumerate}
  \item Coupé
  \item Guardia
  \item Defensa
  \item Contra-ataque
  \item Finta
  \item Provocar
  \item Engañar
  \item Distraer
  \item Llamada
  \item Finta-Pase
  \item Reflejos
  \item Capacidad defensiva
  \item Capacidad ofensiva
  \item Confianza
  \item Explosividad
\end{enumerate}

También ampliamos las tablas de objeto atributo-valor. Ver cuadro 5.4

\begin{table}[]
  \centering
  \caption{Tabla objeto atributo y valor}
  \label{tab:Tabla objeto atributo y valor}
  \begin{tabular}{lll}
    Objeto & Atributo & Valor \\ \hline
    \multicolumn{1}{l|}{\multirow{4}{*}{Tirador}} & Confianza & \{Si, No\} \\ \cline{2-3}
    \multicolumn{1}{l|}{} & Reflejos & \{Bajo, Medio, Alto\} \\ \cline{2-3}
    \multicolumn{1}{l|}{} & Velocidad & \{Bajo, Medio, Alto\} \\ \cline{2-3}
    \multicolumn{1}{l|}{} & Capacidad defensiva & \{Bajo, Medio, Alto\} \\ \cline{2-3}
    \multicolumn{1}{l|}{} & Capacidad ofensiva & \{Bajo, Medio, Alto\} \\ \cline{2-3}
  \end{tabular}
\end{table}


El glosario final se puede ver en el anexo D.
La tabla de objeto, atributo y valor final se puede ver en el anexo E.
