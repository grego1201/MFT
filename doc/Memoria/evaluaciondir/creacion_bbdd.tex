\section{Base de datos}

Ante la nula información almacenada de una forma estructurada se ha tenido que buscar
 diversas maneras de obtener y almacenar la información, estructurarla y tratarla para
 poder sacar conocimiento de ella.

En la página de la federación internacional de esgrima (FIE) se almacenan los resultados
 de las competiciones de los últimos tiempos. De ahí se puede obtener los resultados
 de cada competición tanto como del ranking general, pasando por la fase de poules
 y acabando con los tablones eliminatorios de estos mismos. Viendo toda esta información
 almacenada se decide extraer y guardar dicha información. Para la extracción se
 utilizarán técnicas de scrapping web mediante la cual se descarga la página y se
 obtiene su contenido para tratarlo. En este caso navegamos por dicha página y una
 vez obtenida la información la guardamos en nuestra BBDD. Puesto que no está del
 todo estructurada tendremos que ir almacenando dicha información en diferentes BBDD
 y después juntarlas.

\subsection{Obtención BBDD}

La primera BBDD que generaremos será aquella en la que guardaremos la información
 básica de los asaltos. Para ello guardaremos el ID de la competición, el número
 de tablón en el que se efectuó el asalto, ID del primer tirador, ID del segundo
 tirador, tocados obtenidos por el primer tirador y tocados obtenidos por
 el segundo tirador.

\begin{table}[htb]%
  \centering
  \caption{Estructura BBDD asaltos inicial}
  \label{tab:anchura}
  \begin{tabular}{ | l | l | l | l | l | l | }
    \hline
    Nombre de Campo & Tipo de campo & Ejemplo \\ \hline
    CompetitionID & Texto & 2019-64 \\ \hline
    Tableu & Entero & 32 \\ \hline
    Competitor1 & Texto & /fencers/Anna-KOROLEVA-40351/ \\ \hline
    Competitor2 & Texto & /fencers/Kira-KESZEI-49034/ \\ \hline
    ResultCompetitor1 & Texto & V/15 \\ \hline
    ResultCompetitor2 & Texto & D/3 \\
    \hline
  \end{tabular}
\end{table}

El siguiente paso que tendremos que dar será obtener la información de los tiradores
 para ello ser visitará la página correspondiente. Un ejemplo de página que almacena
 la información de un tirador sería el siguiente http://fie.org/es/fencers/Mario-PERSU-31870
 como se puede observar el final de la URL es el mismo que el identificador almacenado en la
 anterior tabla. De modo que explorando la anterior BBDD podremos visitar las páginas de cada
 tirador almacenado y de ese modo generar una nueva BBDD con toda la información de cada uno de ellos.
 En dicha BBDD tendremos su identificador, edad, ranking (si lo tuvieran), nacionalidad, mano usada
 y arma.

\begin{table}[htb]%
  \centering
  \caption{Estructura BBDD tiradores}
  \label{tab:anchura}
  \begin{tabular}{ | l | l | l | l | l | l | }
    \hline
    Nombre de Campo & Requerido & Tipo de campo & Ejemplo \\ \hline
    competitorID & Si & Texto & ADRIANA-MILANO-36467 \\ \hline
    Age & Si & Entero & 21 \\ \hline
    FieRanking & No & Entero & 300 \\ \hline
    HandNess & Si & Texto & Right \\ \hline
    Weapon & Si & Texto & Epée \\
    \hline
  \end{tabular}
\end{table}

Una vez obtenidos todos los datos referentes a los tiradores tendremos que cruzar
 las dos tablas mencionadas anteriormente de modo que tengamos toda la información
 en una sola BBDD. Esta última BBDD tendrá la información de la primera, sustituyendo
 las columnas de ID de cada tirador por su registro correspondiente en la anterior tabla.

De modo que la estructura será la siguiente
\begin{table}[htb]%
  \centering
  \caption{Estructura BBDD asaltos}
  \label{tab:anchura}
  \begin{tabular}{ | l | l | l | l | l | l | }
    \hline
    Nombre de Campo & Tipo de campo & Ejemplo \\ \hline
    ComptetitionID & Texto & 2019-176 \\ \hline
    TABLEU & Entero & 32 \\ \hline
    C1\_ID & Texto & Sergey-KHODOS-13869 \\ \hline
    C1\_RANKING & Entero & 22 \\ \hline
    C1\_NATIONALITY & Texto & RUSSIA \\ \hline
    C1\_HANDNESS & Texto & Right \\ \hline
    C1\_WEAPON & Texto & Epée \\ \hline
    C2\_ID & Texto & Laurin-EGGENSCHWILER-5966 \\ \hline
    C2\_RANKING & Entero & 34 \\ \hline
    C2\_NATIONALITY & Texto & SWITZERLAND \\ \hline
    C2\_HANDNESS & Texto & Right \\ \hline
    C2\_WEAPON & Texto & Epée \\ \hline
    C2\_ID & Texto & Laurin-EGGENSCHWILER-5966 \\ \hline
    C1\_RESULT & Texto & V/15 \\ \hline
    C2\_RESULT & Texto & D/3 \\
    \hline
  \end{tabular}
\end{table}

\newpage

Después de todo esto habría que tratar de preprocesar los datos para que sean
 lo mas fáciles de tratar posibles. Para ello se decidió coger aquellos campos
 de tipo texto que se pudieran transformar a enteros directamente y se modificaron.

Para ello se ha decido transformar la mano del tirador en un booleano, donde
 0 indica izquierda y 1 derecha. También se ha transformado el tipo de arma
 tal que 0 es espada, 1 florete y 2 sable. De este modo hemos conseguido el mayor
 número de campos tipo número posible. Mas adelante se vió como se podía tratar el campo de resultado de manera que se
 eliminara la descripción que daba sobre si fue victoria o derrota quedandose
 solo con los puntos alcanzados. Para indicar quien fue el ganador del encuentro
 se creó un nuevo campo llamado "WINNER" el cual será de tipo entero y almacenará un
 1 en caso de que haya ganado el tirador 1 y un 2 en caso de que haya ganado el tirador 2.
 En la siguiente tabla tenemos la correspondencia.

\begin{table}[htb]%
  \centering
  \caption{Tabla correspondencia transformación valores BBDD}
  \label{tab:anchura}
  \begin{tabular}{ | l | ll | }
    \hline
    Campo & \multicolumn{1}{l|}{Valor antiguo} & Valor nuevo \\ \hline
    Handness & Left & 0 \\ \cline{2-3}
     & Right & 1 \\ \hline
    Weapon & Epee & 0 \\ \cline{2-3}
     & Foil & 1 \\ \cline{2-3}
     & Sabre & 2 \\ \hline
  \end{tabular}
\end{table}

\subsection{Descripción BBDD}

Con estos últimos cambios la BBDD quedaría tal que:
\begin{table}[htb]%
  \centering
  \caption{Estructura BBDD asaltos}
  \label{tab:anchura}
  \begin{tabular}{ | l | l | l | l | l | l | }
    \hline
    Nombre de Campo & Tipo de campo & Ejemplo \\ \hline
    ComptetitionID & Texto & 2019-176 \\ \hline
    TABLEU & Entero & 32 \\ \hline
    C1\_ID & Texto & Sergey-KHODOS-13869 \\ \hline
    C1\_RANKING & Entero & 22 \\ \hline
    C1\_NATIONALITY & Texto & RUSSIA \\ \hline
    C1\_HANDNESS & Texto & Right \\ \hline
    C1\_WEAPON & Texto & Epée \\ \hline
    C2\_ID & Texto & Laurin-EGGENSCHWILER-5966 \\ \hline
    C2\_RANKING & Entero & 34 \\ \hline
    C2\_NATIONALITY & Texto & SWITZERLAND \\ \hline
    C2\_HANDNESS & Entero & 0 \\ \hline
    C2\_WEAPON & Entero & 0 \\ \hline
    C1\_RESULT & Entero & 15 \\ \hline
    C2\_RESULT & Entero & 3 \\ \hline
    WINNER & Entero & 1 \\
    \hline
  \end{tabular}
\end{table}

Significando cada uno de sus campos lo siguiente:
\begin{itemize}
  \item \textbf{CompetitionID:} Identificador de la competición.
  \item \textbf{TABLEU:} Tablón (ronda) en el que se disputaba el asalto.
  \item \textbf{C1\_ID:} Identificador mediante el cual podremos buscar al tirador
    número uno del asalto.
  \item \textbf{C1\_RANKING:} Ranking internacional del primer tirador del asalto.
  \item \textbf{C1\_NATIONALITY:} Nacionalidad del primer tirador del asalto.
  \item \textbf{C1\_HANDNESS:} Mano usada por el primer tirador del asalto. Siendo
    0 el equivalente a diestro mientras que 1 el equivalente a zurdo.
  \item \textbf{C1\_WEAPON:} Mano usada por el primer tirador del asalto. Siendo 0
    el equivalente a espada, 1 a florete y 2 a sable.
  \item \textbf{C2\_ID:} Identificador mediante el cual podremos buscar al tirador
    número dos del asalto.
  \item \textbf{C2\_RANKING:} Ranking internacional del segundo tirador del asalto.
  \item \textbf{C2\_NATIONALITY:} Nacionalidad del segundo tirador del asalto.
  \item \textbf{C2\_HANDNESS:} Mano usada por el segundo tirador del asalto. Siendo
    0 el equivalente a diestro mientras que 1 el equivalente a zurdo.
  \item \textbf{C2\_WEAPON:} Mano usada por el segundo tirador del asalto. Siendo 0
    el equivalente a espada, 1 a florete y 2 a sable.
  \item \textbf{C1\_RESULT:} Puntos obtenidos por el primer tirador del asalto.
  \item \textbf{C2\_RESULT:} Puntos obtenidos por el segundo tirador del asalto.
  \item \textbf{WINNER:} Marca el ganador del asalto, siendo 0 el primero y 1 el segundo.
\end{itemize}

\newpage

\subsection{Modo de empleo}

El modo de empleo de esta BBDD será para generar un arbol de decisión con el
  cual se podrá hacer un pronóstico. De esta manera el sistema experto tendrá
  una fuente mas de datos mediante la cual podrá decidir una táctica u otra
  en función de los resultados.
