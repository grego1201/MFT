\section{Características del sistema experto}

%inicio pendiente de revision palabras
%Pendiente de escribir

%fin pendiente de revision palabras

\section{Estudio de viabilidad}
%inicio pendiente de revision palabras
Se procederá a realizar un estudio de viabilidad basado en el test de SLAGEL, este test
 asigna pesos a una serie de características a evaluar divididas en diferentes categorías,
 después de los cálculos pertinentes se obtiene un porcentaje de viabilidad, para obtener
 un porcentaje real se procederá por último a normalizar el resultado obtenido, en el apartado
 anterior puede encontrarse la definición del método seguido para el cálculo del estudio de
 viabilidad.

Para poder entender el significado de las tablas se han usado los siguientes acrónimos
 EX: Expertos TA: Tarea DU: Directivos/Usuarios E: Esencial D: Deseable

\textbf{Características de Plausibilidad}

\begin{table}[htb]%
  \centering
  \caption{Tabla con las características de plausibilidad}
  \label{tab:anchura}
  \begin{tabular}{ | l | l | l | l | p{8cm} | l | }
    \hline
    Cat. & Iden & Peso & Valor & Denominación & Tipo \\ \hline
    EX & P1 & 10 & 10 & Existen Expertos & E \\ \hline
    EX & P2 &  10 & 9 & El experto asignado es genuino & E \\ \hline
    EX & P3 & 8 & 9 & El experto es cooperativo & D \\ \hline
    EX & P4 & 7 & 8 & El experto es capaz de articular sus métodos pero no categoriza & D \\ \hline
    TA & P5 & 10 & 9 & Existen suficientes casos de prueba; normales, típicos, ejemplares, correosos, etc & E \\ \hline
    TA & P6 & 10 & 9 & La tarea está bien estructurada y se entiende & D \\ \hline
    TA & P7 & 10 & 9 & Solo requiere habilidad cognoscitiva (no pericia física) & D \\ \hline
    TA & P8 & 9 & 8 & No se precisan resultados óptimos sino sólo satisfactorios, sin comprometer el proyecto & D \\ \hline
    TA & P9 & 9 & 7 & La tarea no requiere sentido común & D \\ \hline
    DU & P10 & 7 & 9 & Los directivos están verdaderamente comprometidos con el proyecto & D \\ \hline
  \end{tabular}
\end{table}

\textbf{Fundamentos de plausibilidad}

A continuación se fundamentan algunos de los valores elegidos para las características de Plausibilidad


\begin{compactitem}
  \item[\textbf{P1}:] Actualmente se dispone de muchos expertos en el sector. Toda sala de esgrima
     tiene un maestro el cual es un experto, con mayor o menor experiencia, el cual transmite
     sus conocimientos adquiridos con los años y los sucesos que vivió a sus alumnos. Por lo tanto
     podríamos decir que hay al menos un experto por sala de esgrima.
  \item[\textbf{P3}:] El experto escogido tiene especial interés en el proyecto, puesto que
     serviría de gran ayuda para sus alumnos en competiciones dado que actualmente es el único
     en poder dar apoyo a estos en esas situaciones.
  \item[\textbf{P7}:] Unicamente se requiere el conocimiento suficiente y experiencia en competición
     para poder identificar las acciones del rival para poder decidir que acciones llevar a cabo
     de manera que se contrarresten las del rival.
  \item[\textbf{P9}:] Al ser una serie de casos con unas entradas y salidas bien definidas, no requiere
     de un gran ingenio poder llevar a cabo la decisión, una vez tengamos todos los casos, o el mayor
     número de estos posibles, identificados.
\end{compactitem}


\textbf{Características de justificación}
\begin{table}[htb]%
  \centering
  \caption{Tabla con las características de justificación}
  \label{tab:anchura}
  \begin{tabular}{ | l | l | l | l | p{8cm} | l | }
    \hline
    Cat. & Iden & Peso & Valor & Denominación & Tipo \\ \hline
    EX & J1 & 10 & 9 & El experto no esta disponible & E \\ \hline
    EX & J2 & 10 & 8 & Hay escasez de experiencia humana & D \\ \hline
    TA & J3 & 8 & 9 & Existe la necesidad de experiencia simultánea en muchos lugares & D \\ \hline
    TA & J4 & 10 & 7 & Necesidad de experiencia en entornos hostiles, penosos y/o poco gratificantes & D \\ \hline
    TA & J5 & 8 & 9 & No existen soluciones alternativas admisibles & E \\ \hline
    DU & J6 & 10 & 9 & Se espera una alta tasa de recuperación de la inversión & D \\ \hline
    DU & J7 & 10 & 9 & Resuelve una tarea útil y necesaria & E \\ \hline
   \end{tabular}
\end{table}

\textbf{Fundamentos de justificación}
A continuación se fundamentan algunos de los valores elegidos para las características de justificación

\begin{compactitem}
  \item[\textbf{J1}:] En competiciones, sobre todo regionales y clubes pequeños, el experto
     no suele estar disponible puesto que en la mayoría de las ocasiones tiene otras labores
     como directorio técnico o incluso ser el mismo un partipante mas de la competición.
     En el mejor de los casos de que no tenga ninguan de estas labores lo normal será que
     tenga a varios alumnos que atender a la vez, por lo que será una situación común que no esté libre.
  \item[\textbf{J3}:] Se puede dar el caso de que dos alumnos de un mismo maestro tengan un
     asalto en el mismo instante. Este no podrá estar en ambos sitios a la vez y tampoco es
     aconsejable estar poco tiempo en uno, después ir al otro y así sucesivamente, por lo que
     se ve la necesidad de este conocimiento en el mismo instante en distintos lugares.
  \item[\textbf{J7}:] Al resolver la tarea de las incertidumbres sobre que hacer
     en cada una de las situacione será mas accesible el deporte para aquellos que estén
     empezando, puesto que no generará esos sentimientos de frustración por no saber que hacer.
\end{compactitem}
\newpage

\textbf{Características de adaptación}
\begin{table}[htb]%
  \centering
  \caption{Tabla con las características de adaptación}
  \label{tab:anchura}
  \begin{tabular}{ | l | l | l | l | p{8cm} | l | }
    \hline
    Cat. & Iden & Peso & Valor & Denominación & Tipo \\ \hline
    EX & A1 & 5 & 8 & La experiencia del experto está poco organizada & D \\ \hline
    TA & A2 & 6 & 9 & Tiene valor práctico & D \\ \hline
    TA & A3 & 7 & 9 & Es una tarea más táctica que estratégica & D \\ \hline
    TA & A4 & 7 & 10 & La tarea da soluciones que sirvan de necesidades a largo plazo & E \\ \hline
    TA & A5 & 5 & 8 & La tarea no es demasiado fácil, pero es de conocimiento intensivo, tanto propio del dominio, como de manipulación de la información & D \\ \hline
    TA & A6 & 6 & 9 & Es de tamaño manejable, y/o es posible un enfoque gradual y/o, una descomposición en subtareas independientes & D \\ \hline
    EX & A7 & 7 & 9 & La transferencia de experiencia entre humanos es factible (experto a aprendiz) & E \\ \hline
    TA & A8 & 6 & 6 & Estaba identificada como un problema en el área y los efectos de la introducción de un SE pueden planificarse & D \\ \hline
    TA & A9 & 9 & 8 & No requiere respuestas en tiempo real “Inmediato” & E \\ \hline
    TA & A10 & 9 & 8 & La tarea no requiere investigación básica & E \\ \hline
    TA & A11 & 5 & 8 & El experto usa básicamente razonamiento simbólico que implica factores subjetivos & D \\ \hline
    TA & A12 & 5 & 8 & Es esencialmente de tipo heurístico & D \\ \hline
  \end{tabular}
\end{table}

\textbf{Fundamentos de adaptación}
A continuación se fundamentan algunos de los valores elegidos para las características de adaptación

\begin{compactitem}
  \item[\textbf{A1}:] Actualmente el experto no tiene ningún sistema en el que
     se pueda consultar su experiencia, no hay nada documentado por lo tanto no
     hay organización alguna.
  \item[\textbf{A4}:] En este caso el sistema no solo sirve para ayudar en el
     instante que se consulta, si no que también sirve para transmitir dicho
     conocimiento al deportista, logrando así una mayor comunidad con
     conocimiento básico sobre el deporte. De este modo con el paso del tiempo
     será mas fácil que el conocimiento se pueda expandir
  \item[\textbf{A6}:] Puesto que los ataques pueden ser compuestos, se podrán
     hacer enfoques graduales en los que se lleven a cabos pensamientos y
     acciones con mayor profundidad, pudiendo dar estos lugar a acciones mas
     complejas. De igual manera se podrá hacer de una manera mas sencilla
     en función de las cualidades del tirador.
  \item[\textbf{A7}:] Es algo tan factible como que se lleva haciendo durante
     mucho tiempo, puesto que son los maestros de esgrima (expertos) quienes
     pasan su experiencia a sus alumnos a diario en las clases que se imparten.
  \item[\textbf{A9}:] Antes de empezar un asalto de esgrima se ha de tener
     clara la táctica a seguir, por lo que no serviría de nada reinventarse
     en mitad del asalto. Por lo tanto se puede llegar a la conclusión de que
     no es necesaria una respuesta inmediata ya que entre asaltos como mínimo
     hay un minuto de descanso, tiempo mas que suficiente para obtener una respuesta.
  \item[\textbf{A11}:] Algunas de las características que se comparan entre
     tiradores son totalmente objetivas, como la altura, pero otras como la experiencia
     la rapidez y la frialdad serán cosas subjetivas que se han de percibir.
\end{compactitem}
\newpage

\textbf{Características de éxito}
\begin{table}[htb]%
  \centering
  \caption{Tabla con las características de éxito}
  \label{tab:anchura}
  \begin{tabular}{ | l | l | l | l | p{8cm} | l | }
    \hline
    Cat. & Iden & Peso & Valor & Denominación & Tipo \\ \hline
    EX & E1 & 8 & 9 & No se sienten amenazados por el proyecto, son capaces de sentirse intelectualmente unidos al proyecto & D \\ \hline
    EX & E2 & 6 & 9 & Tienen un brillante historial en la realización de esta tarea.  & D \\ \hline
    EX & E3 & 5 & 6 & Hay acuerdos en lo que constituye una buena solución a la tarea & D \\ \hline
    EX & E4 & 5 & 8 & La única justificación para dar un paso en la solución es la calidad de la solución final & D \\ \hline
    EX & E5 & 6 & 9 & No hay un plazo de finalización estricto, ni ningún otro proyecto depende de esta tarea & D \\ \hline
    TA & E6 & 7 & 10 & No está influenciada por vaivenes políticos & E \\ \hline
    TA & E7 & 8 & 5 & Existen ya SSEE que resuelvan esa o parecidas tareas & D \\ \hline
    TA & E8 & 8 & 7 & Hay cambios mínimos en los procedimientos habituales & D \\ \hline
    TA & E9 & 5 & 9 & Las soluciones son explicables o interactivas & D \\ \hline
    TA & E10 & 7 & 8 & La tarea es de I+D de carácter práctico, pero no ambas cosas simultáneamente.  & E \\ \hline
    DU & E11 & 6 & 8 & Están mentalizados y tienen expectativas realistas tanto en alcance como en las limitaciones & D \\ \hline
    DU & E12 & 7 & 9 & No rechazan de plano esta tecnología & E \\ \hline
    DU & E13 & 6 & 8 & El sistema interactúa inteligente y amistosamente con el usuario & D \\ \hline
    DU & E14 & 9 & 9 & El sistema es capaz de explicar al usuario su razonamiento & D \\ \hline
    DU & E15 & 8 & 9 & La inserción del sistema se efectúa sin traumas; es decir, apenas se interfiere en la rutina cotidiana de la empresa & D \\ \hline
    DU & E16 & 6 & 9 & Están comprometidos durante toda la duración del proyecto, incluso después de su implementación & D \\ \hline
    DU & E17 & 8 & 8 & Se efectúa una adecuada transferencia tecnológica & E \\ \hline
  \end{tabular}
\end{table}

\textbf{Fundamentos de éxito}
A continuación se fundamentan algunos de los valores elegidos para las características de éxito.

\begin{compactitem}
  \item[\textbf{E1}:] La idea de llevar a cabo este proyecto fue totalmente respaldada
     por el experto una vez que se comentó, involucrandose y formando parte de él desde
     el primer momento.
  \item[\textbf{E9}:] Todas las soluciones se pueden explicar argumentando los motivos
     que da el experto por las que fueron tomadas, de tal manera que el usuario sea
     capaz de entenderlas.
\end{compactitem}

\newpage
\textbf{Resulado}

En la siguiente tabla se muestra el resultado de la evaluación de las diferentes dimensiones
 siguiendo las fórmulas enunciadas en el test de SLAGEL. Una vez evaluadas dichas dimensiones
 se obtiene la media y se normaliza el valor, es decir, se expresa en tanto por ciento.

\begin{table}[htb]%
  \centering
  \caption{Resultados de viabilidad}
  \label{tab:anchura}
  \begin{tabular}{ | l | l | l | l | p{1.5cm} | p{1.5cm} | }
    \hline
    Característica & $\pi\text{(Valor total)}$ & $\pi\text{(Peso total)}$ & Resultado & Resultado VC & Resultado máximo \\ \hline \hline
    Plausibilidad & $3.1752\text{e}9$ & $2.38085568\text{e}9$ & $(7.559692955\text{e}18)^{1/10}$ & 77.24 & 86.63 \\ \hline
    Justificación & $3.584\text{e}6$  & $3.31\text{e}6$ & ($1.18\text{e}13)^{1/7}$ & 73.73 & 85.37 \\ \hline
    Adecuación & $3.75\text{e}9$ & $1.03\text{e}11$ & $(3.87\text{e}20)^{1/12}$ & 51.95 & 82.75 \\ \hline
    Éxito & $9.83\text{e}13$ & $2.96\text{e}15$ & $(2.91\text{e}29)^{1/17}$ & 54.59 & 81.30 \\ \hline \hline
    \multicolumn{4}{|l|}{VC Total} & \multicolumn{2}{l|}{64.38} \\ \hline
    \multicolumn{4}{|l|}{VC Normalizado} & \multicolumn{2}{l|}{76.63} \\ \hline

  \end{tabular}
\end{table}

\textbf{Conclusión}

El porcentaje obtenido en la evaluación es suficiente como para seguir adelante
 con el proyecto, además sºi normalizamos el porcentaje sube hasta el 76.63\%, porcentaje
 mas que suficiente para confiar en la viabilidad del proyecto.

%fin pendiente de revision palabras


