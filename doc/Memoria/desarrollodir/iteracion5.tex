\section{Iteración 5}

%Aquí va la ampliación con IA, BBDD y análisis de datos

Para ampliar el conocimiento y dar un mayor abaníco de opciones
a tener en cuenta en nuestro sistema de apoyo a la toma de decisión
se intenta pronosticar quien será el ganador de un asalto, de esa forma
el sistema podrá tenerlo en cuenta para dar una recomendación u otra.


Para el pronóstico se usarán técnicas de aprendizaje automático que se
verán mas adelante. Para poder aplicarlas hace falta una BBDD. Después de
una larga búsqueda fallida se decide crear nuestra propia BBDD sacando
la información de la página de la federación internacional de esgrima (FIE).

\subsection{Obtención BBDD}

En dicha página disponemos de toda la información de cada torneo acontecido
desde hace varios años con los resultados obtenidos en cada encuentro. Gracias
a esto podemos recopilar toda la información necesaria para poder obtener conocimiento
y transmitirlo a nuestro sistema para que este pueda nutrirse de él y que de este
modo sea mas amplio y fiable.

Para la obtención de datos se han realizado diferntes scripts desarrollados en
python mediante la técnica de scrapping, por lo que es posible que no sean funcionales
cuando esté leyendo esto dado que la página podría haber cambiado el diseño y/o
estructura, pero tan solo habría que adaptar parte de ellos para que vuelvan a
ser funcionales. La técnica de scrapping se basa en visitar la página web y
explorar su código para obtener la información de la misma.

Primero observamos la página y vimos como estaba estructurada. En este caso listaban
los torneos dando información sobre ellos como el tipo de torneo, género, arma, categoría,
etc. También usaban una paginación por lo que esto nos hacía pensar que había muchos tipos
de torneos y desde hace bastantes años. Cuando nos metimos a ver la estructura de dichos torneos
vimos como los registros eran diferentes, indicando solo la clasificación general algunos de ellos.
Esto sucedía sobre todo cuanto mas tiempo tuvieran, por lo que se decidió hacer una primera
criba y obtener aquellos a partir del 2015, que son aquellos en los que se empezaron a registrar
los resultados de las eliminatorias además de la clasificación general del torneo. Esta es la
información que nos interesa sacar, dado que con la clasificación general será algo más
complicado sacar conocimiento de esta para nuestro objetivo final. Mientras que con los
resultados obtenidos de cada enfrentamiento, junto a las características de cada tirador
como pueden ser ranking FIE, mano usada, edad y nacionalidad podremos sacar algo de
conocimiento con todo ello.

Con la primera criba hecha pasamos a sacar la información de cada torneo. Para ello exploraremos
la página con la información de cada competencia la cual contiene la clasificación general,
la cual no nos interesa almacenar. Por otro lado tenemos la fase de grupos o también
conocida como poules. Esta fase también la ignoraremos de momento pudiendo ser esta
información a analizar en un futuro. Ya sólo nos quedarían las fases eliminatorias o
cuadros de directas. Estos contienen el resultado del enfrentamiento: tocados dados por cada tirador,
quien ganó, el identificador de cada tirador y el tablón que se disputaba. Con esto la estructura de la BBDD
se queda de la siguiente forma (ver tabla 5.7):

\begin{table}[]
  \centering
  \caption{Estructura BBDD Inicial}
  \label{tab:Estructura BBDD Inicial}
  \begin{tabular}{|llll|}
    \hline \rowcolor[HTML]{C0C0C0}
    Campo & Tipo & Descripción & Ejemplo \\ \hline
    CompetitionID & String & Identificador de la competición & 2019-64 \\ \hline
    Tableu & Integer & 32 & 32 \\ \hline
    Competitor1 & String & Identificador del competidor 1 & /fencers/Anna-KOROLEVA-40351/ \\ \hline
    Competitor2 & String & Identificador del competidor 2 & /fencers/Kira-KESZEI-49034/ \\ \hline
    ResultCompetitor1 & String & Resultado del competidor 1 & V/15 \\ \hline
    ResultCompetitor2 & String & Resultado del competidor 2 & D/13 \\ \hline
  \end{tabular}
\end{table}

A continuación en la tabla 5.8 se puede ver un ejemplo del estado inicial de la BBDD.

\begin{table}[]
  \centering
  \caption{Ejemplo BBDD inicial}
  \label{tab:Ejemplo BBDD inicial}
  \begin{tabular}{|llllll|}
    \hline \rowcolor[HTML]{C0C0C0}
    CompetitionID & Tableu & Competitor1 & Competitor2 & ResultCompetitor1 & ResultCompetitor2 \\ \hline
    2019-64 & 32 & /fencers/Anna-KOROLEVA-40351/ & /fencers/Kira-KESZEI-49034/ & V/15 & D/13 \\ \hline
    2019-64 & 32 & /fencers/Greta-CECERE-45345/ & /fencers/Andreea-LUPU-37410/ & V/15 & D/12 \\ \hline
  \end{tabular}
\end{table}

Con esta información recopilada nos faltaría completarla. Para ello extraimos en una BBDD
aparte todos los identificadores de los competidores, de esa forma podríamos extraer
sus características en una tabla aparte la cual recopile la información de todos
los competidores. Esta tabla contendrá la siguiente información de cada tirador:
identificador, edad, ranking, nacionalidad, mano dominante y arma. De tal forma
que la estructura se puede ver en la tabla 5.9 y un ejemplo en la tabla 5.10.

\begin{table}[]
  \centering
  \caption{Estructura BBDD tiradores}
  \label{tab:Estructura BBDD tiradores}
  \begin{tabular}{|llll|}
    \hline \rowcolor[HTML]{C0C0C0}
    Campo & Tipo & Descripción & Ejemplo \\ \hline
    ID & String & Identificador del tirador & ADRIANA-MILANO-36467 \\ \hline
    Edad & Integer & Edad del tirador & 32 \\ \hline
    FieRanking & String & Ranking de campeonatos FIE del tirador & 1234 \\ \hline
    Nacionalidad & String & Nacionalidad del tirador & VENEZUELA \\ \hline
    Mano & String & Mano dominante del tirador & Right \\ \hline
    Arma & String & Arma principal del tirador & Sabre \\ \hline
  \end{tabular}
\end{table}

\begin{table}[]
  \centering
  \caption{Ejemplo BBDD tiradores}
  \label{tab:Ejemplo BBDD tiradores}
  \begin{tabular}{|llllll|}
    \hline
    \rowcolor[HTML]{C0C0C0}
    ID & Edad & FieRanking & Nacionalidad & Mano & Arma \\ \hline
    ADRIANA-MILANO-36467 & 21 & None & VENEZUELA & Right & Sabre \\ \hline
    AKHMETOV-Iskander-35108 & 22 & 82 & RUSSIA & Right & Foil \\ \hline
  \end{tabular}
\end{table}

Con toda la información recopilada juntarmeos ambas BBDD de forma que
cruzando los datos de ambas podremos sustituir los identificadores del
de los tiradores de la primera BBDD con los datos recopilados de cada tirador.

De esta forma conseguiremos tener toda la información en una sola BBDD para
que los datos puedan ser comparados con mayor facilidad. La estructura final
de la BBDD se puede ver en la tabla 5.11

\begin{table}[]
  \centering
  \caption{Estructura BBDD final}
  \label{Estructura BBDD final}
  \begin{tabular}{|llll|}
    \hline \rowcolor[HTML]{C0C0C0}
    Campo & Tipo & Descripción & Ejemplo \\ \hline
    ID & String & Identificador de la competición & ADRIANA-MILANO-36467 \\ \hline
    TABLEU & Integer & Tablón en el que se juega el asalto & 32 \\ \hline

    \rowcolor[HTML]{969696}
    C1\_ID & String & Identificador del primer tirador & Alexander-CHOUPENITCH-21765 \\ \hline
    \rowcolor[HTML]{969696}
    C1\_AGE & Integer & Edad del primer tirador & 32 \\ \hline
    \rowcolor[HTML]{969696}
    C1\_RANKING & Integer & Ranking del primer tirador & 10 \\ \hline
    \rowcolor[HTML]{969696}
    C1\_NATIONALITY & String & Nacionalidad del primer tirador & CZECH REPUBLIC \\ \hline
    \rowcolor[HTML]{969696}
    C1\_HANDNESS & String & Mano dominante del primer tirador & Right \\ \hline
    \rowcolor[HTML]{969696}
    C1\_WEAPON & String & Arma principal del primer tirador & Epee \\ \hline

    \rowcolor[HTML]{636363}
    C2\_ID & String & Identificador del segundo tirador & Alexander-CHOUPENITCH-21765 \\ \hline
    \rowcolor[HTML]{636363}
    C2\_AGE & Integer & Edad del segundo tirador & 32 \\ \hline
    \rowcolor[HTML]{636363}
    C2\_RANKING & Integer & Ranking del segundo tirador & 10 \\ \hline
    \rowcolor[HTML]{636363}
    C2\_NATIONALITY & String & Nacionalidad del segundo tirador & CZECH REPUBLIC \\ \hline
    \rowcolor[HTML]{636363}
    C2\_HANDNESS & String & Mano dominante del segundo tirador & Right \\ \hline
    \rowcolor[HTML]{636363}
    C2\_WEAPON & String & Arma principal del segundo tirador & Epee \\ \hline

    \rowcolor[HTML]{969696}
    RESULT\_C1 & String & Resultado del primer tirador & V/15 \\ \hline
    \rowcolor[HTML]{636363}
    RESULT\_C2 & String & Resultado del segundo tirador & D/7 \\ \hline

  \end{tabular}
\end{table}

\subsection{Tratamiento BBDD}

El primer paso que daremos será añadir una columna numérica indicando quien
fue el ganador del encuentro. Esta contendrá 0 o 1 dependiendo de si ganó
el primer tirador o el segundo respectivamente. De esta forma será mas fácil
identificar quien de los dos tiradores ganó. La estructura del nuevo campo
se podrá ver en la tabla 5.12. Para la obtención de este campo se exploró
el asalto correspondiente y se comprobó quien tenía dentro de su resultado
la letra \"V\" dado que esta es la que indicaba la victoria.

\begin{table}[]
  \centering
  \caption{Estructura campo ganador}
  \label{tab:Estructura campo ganador}
  \begin{tabular}{|llll|}
    \hline \rowcolor[HTML]{C0C0C0}
    Campo & Tipo & Descripción & Ejemplo \\ \hline
    WINNER & Integer & Este campo nos indicará quien ganó. Siendo 0 el primer tirador y 1 el segundo & 1 \\ \hline
  \end{tabular}
\end{table}

Una vez calculado este campo ya no nos harían falta los campos de resultados
por lo que se procedió al borrado de los mismos.

El siguiente cambio que harémos será normalizar la BBDD. Esto se hace para
que los datos puedan ser comparados de una forma mas eficiente. Para ello
transformamos aquellas variables de texto a numéricas, de este modo serán
mas sencillos los cálculos estadísticos.

En nuestro caso tenemos la mano usada por el tirador y el arma usada por este.
Para ello seguiremos la transformación mostrada en la taba 5.13 y 5.14 para
las equivalencias de mano y arma respectivamente. Esto se resume en
cambiar la mano dominante diestra por un valor de 0 mientras que la zurda será
sustituida por el valor 1. También cambiaremos las armas siendo el equivalente
de espada el 2, 1 para florete y restando el 0 para sable.

\begin{table}[]
  \centering
  \caption{Tabla equivalencias mano}
  \label{tab:Tabla equivalencias mano}
  \begin{tabular}{|ll|}
    \hline
    \rowcolor[HTML]{C0C0C0}
    Valor antiguo & Valor nuevo \\ \hline
    Right & 0 \\ \hline
    Left & 1 \\ \hline
  \end{tabular}
\end{table}

\begin{table}[]
  \centering
  \caption{Tabla equivalencias mano}
  \label{tab:Tabla equivalencias mano}
  \begin{tabular}{|ll|}
    \hline
    \rowcolor[HTML]{C0C0C0}
    Valor antiguo & Valor nuevo \\ \hline
    Sabre & 0 \\ \hline
    Foil & 1 \\ \hline
    Epee & 2 \\ \hline
  \end{tabular}
\end{table}

