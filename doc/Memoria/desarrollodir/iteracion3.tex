\section{Iteración 3}

% Aquí va el desarrollo de la página web

Con el prototipo del sistema experto desarrollado el siguiente paso fue
proporcionar una interfaz amigable para el usuario a la par que accesible
desde cualquier lugar.

Para ello lo primero fue crear unos bocetos sobre el diseño que se quería tener
en la web a papel y bolígrafo. La idea original es que esta se componga de una página
principal en la que se explique el objetivo de la página y cual es su propósito.

Por otro lado tendremos dos secciones que se diferencien la una de la otra de las
cuales una de ellas estará destinada a una versión rápida del sistema de apoyo mientras
que la otra será una versión mas exhaustiva. De este modo nos podremos adaptar
a las diversas situaciones en las que se puede utilizar la herramienta.
La primera será para aquellas situaciones en las que tenemos menos de un minuto
para que nos de un resultado. La segunda será para cuando podamos analizar
los resultados e introducir las variables con mayor calma puesto que no
tendremos prisa para ello.

Los resultados los bocetos son los siguientes:

% Realizar bocetos para TFG

Una vez con los bocetos claros se decidió que tecnologías usar. Para el desarrollo
de la aplicación se decidió usar Ruby puesto que es uno de los lenguajes
de programación mas utilizados para el desarrollo ágil de aplicaciones web.
Otro de los motivos para escoger dicho lenguaje es el conocimiento obtenido
por el desarrollador en dicho lenguaje, de este modo no tendremos que añadir
tiempo de aprendizaje al proyecto, lo cual ahorrará recursos. En este caso el
recurso será el tiempo que tarde en aprender el nuevo lenguaje.

El paradigma utilizado será Modelo-Vista-Controlador. Siendo este el que más
favorece a la mantenebilidad del proyecto, permitiendo la ampliación de requisitos
del mismo mediante modelos para después ampliar a usuarios, perfiles de tiradores,
etc.

El desarrollo de la aplicación web se hizo en local, utilizando las herramientas
que proporciona ruby para ello.

El diagrama de desarrollo será el siguiente:

% Hacer diagrama desarrollo web

\subsection{Inicialización proyecto}

Lo primero será inicializar el proyecto. Para ello se seguirá la guia encontrada
en la página de documentación de ruby. Después añadiremos bootstrap para los estilos.

\subsection{Creación controladores principales}

Crearemos los dos primeros controladores y los mas importantes de la aplicación.
Estos serán los que nos ayuden con la parte guiada (rápida) y por otro lado la
completa. También se agregarán las correspondientes traducciones. También se
agregarán las vistas correspondientes con un pequeño formulario inicial.

\subsection{Creación servicio tomar decisión}

Se creará el servicio que será el encargado de llevar a cabo el sistema experto.
Esto quiere decir que será este servicio el que tendrá toda la lógica de la parte
mas importante de la aplicación. Aquí tendremos que traducir el sistema experto.

\subsection{Creación vista resultados}

Daremos visualización a los resultados obtenidos por el servicio de toma de decisiones.

\subsection{Añadir bootstrap}

Se añadirá bootstrap para facilitar la aplicación de estilos en la página y de esta
manera mejorar la usabilidad de la misma.

\subsection{Añadir traducciones}

Se añadirán las traducciones de todo lo agregado además de preparar la aplicación
para el resto de cambios.

\subsection{Añadir página inicio}

Se añadirá página de inicio

\subsection{Añadir página contacto}

Se añadirá página de contacto

\subsection{Añadir página sugerencias}

Se añadirá página de sugerencias

\subsection{Añadir página entrenamiento}

Se añadirá página de entrenamiento

\subsection{Definir y añadir estilos}

Se definirán los estilos dentro de bootstrap para los formularios, botones, etc.
