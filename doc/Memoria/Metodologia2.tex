\chapter{Metodología}
\label{cap:Metodologia}

Obtención de la BBDD.

Ante la nula información almacenada de una forma estructurada se ha tenido que buscar
 diversas maneras de obtener y almacenar la información, estructurarla y tratarla para
 poder sacar conocimiento de ella.

En la página de la federación internacional de esgrima (FIE) se almacenan los resultados
 de las competiciones de los últimos tiempos. De ahí se puede obtener los resultados
 de cada competición tanto como el ranking general, pasando por la fase de poules
 y acabando con los tablones eliminatorios de estos mismos. Viendo toda esta información
 almacenada se decide extraer y almacenar dicha información. Para la extracción se
 utilizarán técnicas de scrapping web mediante la cual se descarga la página y se
 obtiene su contenido para tratarlo. En este caso navegamos por dicha página y una
 vez obtenida la información la guardamos en nuestra BBDD. Puesto que no está del
 todo estructurada tendremos que ir almacenando dicha información en diferentes BBDD
 y después juntarlas.

Descripción de la BBDD.

La primera BBDD que generaremos será aquella en la que guardaremos la información
 básica de los asaltos. Para ello guardaremos el ID de la competición, el número
 de tablón en el que se efectuó el asalto, ID del primer tirador, ID del segundo
 tirador, tocados obtenidos por el segundo primer tirador y tocados obtenidos por
 el segundo tirador.

\begin{table}[htb]%
  \centering
  \caption{Estructura BBDD asaltos inicial}
  \label{tab:anchura}
  \begin{tabular}{ | l | l | l | l | l | l | }
    \hline
    Nombre de Campo & Tipo de campo & Ejemplo \\ \hline
    CompetitionID & Texto & 2019-64 \\ \hline
    Tableu & Entero & 32 \\ \hline
    Competitor1 & Texto & /fencers/Anna-KOROLEVA-40351/ \\ \hline
    Competitor2 & Texto & /fencers/Kira-KESZEI-49034/ \\ \hline
    ResultCompetitor1 & Texto & V/15 \\ \hline
    ResultCompetitor2 & Texto & D/3 \\
    \hline
  \end{tabular}
\end{table}

El siguiente paso que tendremos que dar será obtener la información de los tiradores
 para ello ser visitará la página correspondiente. Un ejemplo de página que almacena
 la información de un tirador sería el siguiente http://fie.org/es/fencers/Mario-PERSU-31870
 como se puede observar el final de la URL es el mismo que el identificador almacenado en la
 anterior tabla. De modo que explorando la anterior BBDD podremos visitar las páginas de cada
 tirador almacenado y de ese modo generar una nueva BBDD con toda la información de cada uno de ellos.
 En dicha BBDD tendremos su identificador, edad, ranking (si lo tuvieran), nacionalidad, mano usada
 y arma.

\begin{table}[htb]%
  \centering
  \caption{Estructura BBDD tiradores}
  \label{tab:anchura}
  \begin{tabular}{ | l | l | l | l | l | l | }
    \hline
    Nombre de Campo & Requerido & Tipo de campo & Ejemplo \\ \hline
    competitorID & Si & Texto & ADRIANA-MILANO-36467 \\ \hline
    Age & Si & Entero & 21 \\ \hline
    FieRanking & No & Entero & 300 \\ \hline
    HandNess & Si & Texto & Right \\ \hline
    Weapon & Si & Texto & Epée \\
    \hline
  \end{tabular}
\end{table}

Una vez obtenidos todos los datos referentes a los tiradores tendremos que cruzar
 las dos tablas mencionadas anteriormente de modo que tengamos toda la información
 en una sola BBDD. Esta última BBDD tendrá la información de la primera, sustituyendo
 las columnas de ID de cada tirador por su registro correspondiente en la anterior tabla.

De modo que la estructura será la siguiente
\begin{table}[htb]%
  \centering
  \caption{Estructura BBDD asaltos final}
  \label{tab:anchura}
  \begin{tabular}{ | l | l | l | l | l | l | }
    \hline
    Nombre de Campo & Tipo de campo & Ejemplo \\ \hline
    ComptetitionID & Texto & 2019-176 \\ \hline
    TABLEU & Entero & 32 \\ \hline
    C1\_ID & Texto & Sergey-KHODOS-13869 \\ \hline
    C1\_RANKING & Entero & 22 \\ \hline
    C1\_NATIONALITY & Texto & RUSSIA \\ \hline
    C1\_HANDNESS & Texto & Right \\ \hline
    C1\_WEAPON & Texto & Epée \\ \hline
    C2\_ID & Texto & Laurin-EGGENSCHWILER-5966 \\ \hline
    C2\_RANKING & Entero & 34 \\ \hline
    C2\_NATIONALITY & Texto & SWITZERLAND \\ \hline
    C2\_HANDNESS & Texto & Right \\ \hline
    C2\_WEAPON & Texto & Epée \\ \hline
    C2\_ID & Texto & Laurin-EGGENSCHWILER-5966 \\ \hline
    C1\_RESULT & Texto & V/15 \\ \hline
    C2\_RESULT & Texto & D/3 \\
    \hline
  \end{tabular}
\end{table}

Modo de empleo.

El modo de empleo de esta BBDD será para el entrenamiento de una red neuronal
 la cual servirá para apoyar al sistema experto. En total hay unos 28.000 registros,
 de los cuales se emplearán entorno al 60 por ciento para entrenar al modelo y un 40
 por ciento para comprobar que el entrenamiento ha sido satisfactorio.
