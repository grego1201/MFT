\chapter{Test de viabilidad}
\label{cap: Introducción y objetivos}

Este capítulo está dedicado a analizar como de viable es el proyecto. De este modo
se justificará la realización y continuación del mismo. Debido a que se cuenta con
un experto y para realizar el sistema basado en el conocimiento, se llevará a cabo
el estudio de viabilidad mediante el Test de Slagel.

\section{Test de Slagel}


Este test está pensado para Sistemas Expertos. Para la evaluación utilizará una serie
de valores para las características los cuales estarán ponderados de una manera
específica en función de las mismas. Dicho estudio de viabilidad se divide en tres etapas:

\begin{itemize}
  \item Definición de características
  \item Asignación de pesos a cada una de las características
  \item Evaluación de cada aplicación candidata.
\end{itemize}

Las características están divididas en cuatro dimensiones:

\begin{enumerate}
  \item \textbf{Plausibilidad:} En esta dimensión se intentará determinar si se cuenta
  con los medios necesarios desde la perspectiva de la Ingeniería del Conocimiento.
  Para ello se analizan dos aspectos:
  \begin{itemize}
    \item \textbf{Características del experto:} se evaluará su fama dentro del sector.
      Esto quiere decir si los demás profesionales del entorno reconocen su trabajo. También
      se valorará la capacidad de cooperar. Asimismo, otro aspecto a tener en cuenta será
      si es competente articulando sus métodos y procedimientos de trabajo. Otra cuestión
      de suma importancia será que el experto se haya enfrentado al problema con anterioridad
      y que este lo haya resuelto con éxito.
    \item \textbf{Características de la tarea que lleva a cabo el experto} se analizará el
      grado de dificultad que tiene esta. Además se valorará si está adecuadamente estructurada
      y que tipo de habilidades se requieren para su realización.
  \end{itemize}
  \item \textbf{Justificación:} Se puede dar el caso de que el desarrollo de un Sistema Experto
    sea posible, pero esto no quiere decir que esté justificado. En esta dimensión se tratará
    de comprobar la justificación del desarrollo del dicho sistema desde la perspectiva de la
    Ingeniería del Conocimiento. Para ello se analizará lo siguiente:
    \begin{itemize}
      \item \textbf{Necesidad de la experiencia:} se evaluará las características del ambiente
        donde hay que realizar la tarea. Los principales factores a tener en cuenta será
        el peligro que hay en el entorno (terreno hostil), además de la escasez de expertos
        humanos y la necesidad de su presencia en diferentes sitios a la vez.
      \item \textbf{Inversión a realizar:} se tendrá en cuenta los costes que conllevarán
        realizar el Sistema Experto frente al retorno de la inversión realizada. También se
        tendrán en cuenta soluciones alternativas.
      \item \textbf{Transferencia de conocimiento:} un buen motivo para justificar la realización
        del sistema experto sería la posible pérdida de conocimiento. En casos como avanzada
        edad del experto hará que todo el conocimiento adquirido por el mismo pueda perderse.
    \end{itemize}
  \item \textbf{Adecuación: } se estudiará si el problema es adecuado para ser resuelto con técnicas
    de Ingeniería del Conocimiento. Algunos problemas podrán ser resueltos mediante algoritmos convencionales
    o aquellos problemas que requieran de sentido común. Se tendrá en cuenta la naturaleza, complejidad
    y tipo de tarea.
  \item \textbf{Éxito: } se intentarán determinar las posibilidades de éxito del sistema a desarrollar. Para
    ello se tendrán en cuenta como de entrenadas estén las personas implicadas, que el Sistema Experto tenga
    una ubicación idónea, que este sea aceptado por los usuarios como una herramienta de mejora y que este en
    concordancia con sus soluciones junto a la de los expertos.
\end{enumerate}

Se establecerá una categoría de aplicación sobre cada una de estas dimensiones,
mediante las cuales se identificará quien es el destinatario de la tarea.
Se pueden distinguir tres actores: experto, usuario o directivo, o la propia tarea.
Dentro de cada tarea tendremos dos tipos: esenciales o deseables. Las primeras no
podrán tener una puntuación mínima de 7 puesto que como su nombre indica, son esenciales.
El sistema de puntuación será entre 0 y 10 dependiendo de la importancia relativa de la misma.

Teniendo en cuenta lo explicado anteriormente, el proceso de evaluación del proyecto es el siguiente:

\begin{enumerate}
  \item Asignar un valor a cada una de las características en cada dimensión. Dicho valor estará
    entre 0 y 10, cuya escala representará como de presente está respectivamente, siendo 10 totalmente
    presente. En el caso de que una característica esencial obtenga un valor menor de 7 la aplicación
    quedará automáticamente descartada.
  \item Ponderar el valor de la características respecto a su peso.
  \item Multiplicar para cada dimensión los valores ponderados obtenidos anteriormente.
  \item Obtener la media para cada dimensión de los valores ponderados de las características.
    Para ello se calculará la raíz n-ésima del producto obtenido del apartado anterior. Hay que
    emplear como índice el valor máximo de los índices usados en cada dimensión.
  \item Dividir la suma del resultado de cada dimensión entre cuatro (4) pudiendo obtener como máximo
    un valor de 76,21.
\end{enumerate}

A continuación se detalla el cálculo de la viabilidad del TFG haciendo uso del Test de Slagel.
Para mayor comprensión consultar la leyenda de la tabla 2.1.

\begin{table}[]
  \centering
  \caption{Leyenda}
  \label{tab:Leyenda}
  \begin{tabular}{|l|l|l|}
    \hline
    Acrónimo & Significado & Rango \\ \hline
    CAT & Categoría &  \\ \hline
    EX & Experto(s) &  \\ \hline
    TA & Tarea &  \\ \hline
    IDEN. CAR. & Identificador de la característica &  \\ \hline
    Pi & Identificador de la dimensión de Plausibilidad & P1...P10 \\ \hline
    Ji & Identificador de la dimensión de Justificación & J1...J7 \\ \hline
    Ai & Identificador de la dimensión de Adecuación & A1...A12 \\ \hline
    Ei & Identificador de la dimensión de Éxito & E1...E17 \\ \hline
    E & Esencial & 0...10 \\ \hline
    D & Deseable & 0...10 \\ \hline
  \end{tabular}
\end{table}








%inicio pendiente de revision palabras
\subsection{Características de Plausibilidad}

\begin{table}[htb]%
  \centering
  \caption{Tabla con las características de plausibilidad}
  \label{tab:anchura}
  \begin{tabular}{ | l | l | l | l | p{8cm} | l | }
    \hline
    Cat. & Iden & Peso & Valor & Denominación & Tipo \\ \hline
    EX & P1 & 10 & 10 & Existen Expertos & E \\ \hline
    EX & P2 &  10 & 9 & El experto asignado es genuino & E \\ \hline
    EX & P3 & 8 & 9 & El experto es cooperativo & D \\ \hline
    EX & P4 & 7 & 8 & El experto es capaz de articular sus métodos pero no categoriza & D \\ \hline
    TA & P5 & 10 & 9 & Existen suficientes casos de prueba; normales, típicos, ejemplares, correosos, etc & E \\ \hline
    TA & P6 & 10 & 9 & La tarea está bien estructurada y se entiende & D \\ \hline
    TA & P7 & 10 & 9 & Solo requiere habilidad cognoscitiva (no pericia física) & D \\ \hline
    TA & P8 & 9 & 8 & No se precisan resultados óptimos sino sólo satisfactorios, sin comprometer el proyecto & D \\ \hline
    TA & P9 & 9 & 7 & La tarea no requiere sentido común & D \\ \hline
    DU & P10 & 7 & 9 & Los directivos están verdaderamente comprometidos con el proyecto & D \\ \hline
  \end{tabular}
\end{table}

\textbf{Fundamentos de plausibilidad}

A continuación se fundamentan algunos de los valores elegidos para las características de Plausibilidad


\begin{compactitem}
  \item[\textbf{P1}:] Actualmente se dispone de muchos expertos en el sector. Toda sala de esgrima
     tiene un maestro el cual es un experto, con mayor o menor experiencia, el cual transmite
     sus conocimientos adquiridos con los años y los sucesos que vivió a sus alumnos. Por lo tanto
     podríamos decir que hay al menos un experto por sala de esgrima.
  \item[\textbf{P3}:] El experto escogido tiene especial interés en el proyecto, puesto que
     serviría de gran ayuda para sus alumnos en competiciones dado que actualmente es el único
     en poder dar apoyo a estos en esas situaciones.
  \item[\textbf{P7}:] Unicamente se requiere el conocimiento suficiente y experiencia en competición
     para poder identificar las acciones del rival para poder decidir que acciones llevar a cabo
     de manera que se contrarresten las del rival.
  \item[\textbf{P9}:] Al ser una serie de casos con unas entradas y salidas bien definidas, no requiere
     de un gran ingenio poder llevar a cabo la decisión, una vez tengamos todos los casos, o el mayor
     número de estos posibles, identificados.
\end{compactitem}


\subsection{Características de justificación}
\begin{table}[htb]%
  \centering
  \caption{Tabla con las características de justificación}
  \label{tab:anchura}
  \begin{tabular}{ | l | l | l | l | p{8cm} | l | }
    \hline
    Cat. & Iden & Peso & Valor & Denominación & Tipo \\ \hline
    EX & J1 & 10 & 9 & El experto no esta disponible & E \\ \hline
    EX & J2 & 10 & 8 & Hay escasez de experiencia humana & D \\ \hline
    TA & J3 & 8 & 9 & Existe la necesidad de experiencia simultánea en muchos lugares & D \\ \hline
    TA & J4 & 10 & 7 & Necesidad de experiencia en entornos hostiles, penosos y/o poco gratificantes & D \\ \hline
    TA & J5 & 8 & 9 & No existen soluciones alternativas admisibles & E \\ \hline
    DU & J6 & 10 & 9 & Se espera una alta tasa de recuperación de la inversión & D \\ \hline
    DU & J7 & 10 & 9 & Resuelve una tarea útil y necesaria & E \\ \hline
   \end{tabular}
\end{table}

\textbf{Fundamentos de justificación}
A continuación se fundamentan algunos de los valores elegidos para las características de justificación

\begin{compactitem}
  \item[\textbf{J1}:] En competiciones, sobre todo regionales y clubes pequeños, el experto
     no suele estar disponible puesto que en la mayoría de las ocasiones tiene otras labores
     como directorio técnico o incluso ser el mismo un participante mas de la competición.
     En el mejor de los casos de que no tenga ninguna de estas labores lo normal será que
     tenga a varios alumnos que atender a la vez, por lo que será una situación común que no esté libre.
  \item[\textbf{J3}:] Se puede dar el caso de que dos alumnos de un mismo maestro tengan un
     asalto en el mismo instante. Este no podrá estar en ambos sitios a la vez y tampoco es
     aconsejable estar poco tiempo en uno, después ir al otro y así sucesivamente, por lo que
     se ve la necesidad de este conocimiento en el mismo instante en distintos lugares.
  \item[\textbf{J7}:] Al resolver la tarea de las incertidumbres sobre que hacer
     en cada una de las situaciones será mas accesible el deporte para aquellos que estén
     empezando, puesto que no generará esos sentimientos de frustración por no saber que hacer.
\end{compactitem}
\newpage

\subsection{Características de adaptación}
\begin{table}[htb]%
  \centering
  \caption{Tabla con las características de adaptación}
  \label{tab:anchura}
  \begin{tabular}{ | l | l | l | l | p{8cm} | l | }
    \hline
    Cat. & Iden & Peso & Valor & Denominación & Tipo \\ \hline
    EX & A1 & 5 & 8 & La experiencia del experto está poco organizada & D \\ \hline
    TA & A2 & 6 & 9 & Tiene valor práctico & D \\ \hline
    TA & A3 & 7 & 9 & Es una tarea más táctica que estratégica & D \\ \hline
    TA & A4 & 7 & 10 & La tarea da soluciones que sirvan de necesidades a largo plazo & E \\ \hline
    TA & A5 & 5 & 8 & La tarea no es demasiado fácil, pero es de conocimiento intensivo, tanto propio del dominio, como de manipulación de la información & D \\ \hline
    TA & A6 & 6 & 9 & Es de tamaño manejable, y/o es posible un enfoque gradual y/o, una descomposición en subtareas independientes & D \\ \hline
    EX & A7 & 7 & 9 & La transferencia de experiencia entre humanos es factible (experto a aprendiz) & E \\ \hline
    TA & A8 & 6 & 6 & Estaba identificada como un problema en el área y los efectos de la introducción de un SE pueden planificarse & D \\ \hline
    TA & A9 & 9 & 8 & No requiere respuestas en tiempo real “Inmediato” & E \\ \hline
    TA & A10 & 9 & 8 & La tarea no requiere investigación básica & E \\ \hline
    TA & A11 & 5 & 8 & El experto usa básicamente razonamiento simbólico que implica factores subjetivos & D \\ \hline
    TA & A12 & 5 & 8 & Es esencialmente de tipo heurístico & D \\ \hline
  \end{tabular}
\end{table}

\textbf{Fundamentos de adaptación}
A continuación se fundamentan algunos de los valores elegidos para las características de adaptación

\begin{compactitem}
  \item[\textbf{A1}:] Actualmente el experto no tiene ningún sistema en el que
     se pueda consultar su experiencia, no hay nada documentado por lo tanto no
     hay organización alguna.
  \item[\textbf{A4}:] En este caso el sistema no solo sirve para ayudar en el
     instante que se consulta, si no que también sirve para transmitir dicho
     conocimiento al deportista, logrando así una mayor comunidad con
     conocimiento básico sobre el deporte. De este modo con el paso del tiempo
     será mas fácil que el conocimiento se pueda expandir
  \item[\textbf{A6}:] Puesto que los ataques pueden ser compuestos, se podrán
     hacer enfoques graduales en los que se lleven a cabos pensamientos y
     acciones con mayor profundidad, pudiendo dar estos lugar a acciones mas
     complejas. De igual manera se podrá hacer de una manera mas sencilla
     en función de las cualidades del tirador.
  \item[\textbf{A7}:] Es algo tan factible como que se lleva haciendo durante
     mucho tiempo, puesto que son los maestros de esgrima (expertos) quienes
     pasan su experiencia a sus alumnos a diario en las clases que se imparten.
  \item[\textbf{A9}:] Antes de empezar un asalto de esgrima se ha de tener
     clara la táctica a seguir, por lo que no serviría de nada reinventarse
     en mitad del asalto. Por lo tanto se puede llegar a la conclusión de que
     no es necesaria una respuesta inmediata ya que entre asaltos como mínimo
     hay un minuto de descanso, tiempo mas que suficiente para obtener una respuesta.
  \item[\textbf{A11}:] Algunas de las características que se comparan entre
     tiradores son totalmente objetivas, como la altura, pero otras como la experiencia
     la rapidez y la frialdad serán cosas subjetivas que se han de percibir.
\end{compactitem}
\newpage

\subsection{Características de éxito}
\begin{table}[htb]%
  \centering
  \caption{Tabla con las características de éxito}
  \label{tab:anchura}
  \begin{tabular}{ | l | l | l | l | p{8cm} | l | }
    \hline
    Cat. & Iden & Peso & Valor & Denominación & Tipo \\ \hline
    EX & E1 & 8 & 9 & No se sienten amenazados por el proyecto, son capaces de sentirse intelectualmente unidos al proyecto & D \\ \hline
    EX & E2 & 6 & 9 & Tienen un brillante historial en la realización de esta tarea.  & D \\ \hline
    EX & E3 & 5 & 6 & Hay acuerdos en lo que constituye una buena solución a la tarea & D \\ \hline
    EX & E4 & 5 & 8 & La única justificación para dar un paso en la solución es la calidad de la solución final & D \\ \hline
    EX & E5 & 6 & 9 & No hay un plazo de finalización estricto, ni ningún otro proyecto depende de esta tarea & D \\ \hline
    TA & E6 & 7 & 10 & No está influenciada por vaivenes políticos & E \\ \hline
    TA & E7 & 8 & 5 & Existen ya SSEE que resuelvan esa o parecidas tareas & D \\ \hline
    TA & E8 & 8 & 7 & Hay cambios mínimos en los procedimientos habituales & D \\ \hline
    TA & E9 & 5 & 9 & Las soluciones son explicables o interactivas & D \\ \hline
    TA & E10 & 7 & 8 & La tarea es de I+D de carácter práctico, pero no ambas cosas simultáneamente.  & E \\ \hline
    DU & E11 & 6 & 8 & Están mentalizados y tienen expectativas realistas tanto en alcance como en las limitaciones & D \\ \hline
    DU & E12 & 7 & 9 & No rechazan de plano esta tecnología & E \\ \hline
    DU & E13 & 6 & 8 & El sistema interactúa inteligente y amistosamente con el usuario & D \\ \hline
    DU & E14 & 9 & 9 & El sistema es capaz de explicar al usuario su razonamiento & D \\ \hline
    DU & E15 & 8 & 9 & La inserción del sistema se efectúa sin traumas; es decir, apenas se interfiere en la rutina cotidiana de la empresa & D \\ \hline
    DU & E16 & 6 & 9 & Están comprometidos durante toda la duración del proyecto, incluso después de su implementación & D \\ \hline
    DU & E17 & 8 & 8 & Se efectúa una adecuada transferencia tecnológica & E \\ \hline
  \end{tabular}
\end{table}

\textbf{Fundamentos de éxito}
A continuación se fundamentan algunos de los valores elegidos para las características de éxito.

\begin{compactitem}
  \item[\textbf{E1}:] La idea de llevar a cabo este proyecto fue totalmente respaldada
     por el experto una vez que se comentó, involucrandose y formando parte de él desde
     el primer momento.
  \item[\textbf{E9}:] Todas las soluciones se pueden explicar argumentando los motivos
     que da el experto por las que fueron tomadas, de tal manera que el usuario sea
     capaz de entenderlas.
\end{compactitem}

\newpage
\subsection{Resulado}

En la siguiente tabla se muestra el resultado de la evaluación de las diferentes dimensiones
 siguiendo las fórmulas enunciadas en el test de SLAGEL. Una vez evaluadas dichas dimensiones
 se obtiene la media y se normaliza el valor, es decir, se expresa en tanto por ciento.

\begin{table}[htb]%
  \centering
  \caption{Resultados de viabilidad}
  \label{tab:anchura}
  \begin{tabular}{ | l | l | l | l | p{1.5cm} | p{1.5cm} | }
    \hline
    Característica & $\pi\text{(Valor total)}$ & $\pi\text{(Peso total)}$ & Resultado & Resultado VC & Resultado máximo \\ \hline \hline
    Plausibilidad & $3.1752\text{e}9$ & $2.38085568\text{e}9$ & $(7.559692955\text{e}18)^{1/10}$ & 77.24 & 86.63 \\ \hline
    Justificación & $3.584\text{e}6$  & $3.31\text{e}6$ & ($1.18\text{e}13)^{1/7}$ & 73.73 & 85.37 \\ \hline
    Adecuación & $3.75\text{e}9$ & $1.03\text{e}11$ & $(3.87\text{e}20)^{1/12}$ & 51.95 & 82.75 \\ \hline
    Éxito & $9.83\text{e}13$ & $2.96\text{e}15$ & $(2.91\text{e}29)^{1/17}$ & 54.59 & 81.30 \\ \hline \hline
    \multicolumn{4}{|l|}{VC Total} & \multicolumn{2}{l|}{64.38} \\ \hline
    \multicolumn{4}{|l|}{VC Normalizado} & \multicolumn{2}{l|}{84.01} \\ \hline

  \end{tabular}
\end{table}

\textbf{Conclusión}

El porcentaje obtenido en la evaluación es suficiente como para seguir adelante
 con el proyecto, además si normalizamos el porcentaje sube hasta el 84.01\%, porcentaje
 mas que suficiente para confiar en la viabilidad del proyecto.

%fin pendiente de revision palabras
