\chapter{Introducción y objetivos}
\label{cap: Introducción y objetivos}

%En progreso

\section{Resumen}
Pendiente de escribir

\section{Objetivos}

El principal objetivo de este TFG es contribuir al mundo del deporte, en concreto al
 deporte de la esgrima. En el comienzo de este deporte hay una gran curva de aprendizaje
 en cuanto al esquema táctico se refiere puesto que al ser un deporte minoritario los
 recursos que se le dedican son menores por lo que dificulta la expansión de conocimiento
 y por ende, la adquisición de este mismo tanto a personas que ya lo practican como aquellas
 que acaban de empezar. Se quiere reducir la inclinación de dicha curva de aprendizaje
 en el momento que tienes que aprender por ti mismo y necesitas ayuda de los demás
 para saber que acciones son las correctas y porque están mal tomadas algunas decisiones,
 al menos, hasta que te puedes valer por ti mismo como tirador que es capaz de identificar
 las acciones que están ocurriendo y analizar cuales son las mejores decisiones para contrarrestarlas.

Para ello se pretende desarrollar una aplicación la cual sea capaz de llevar a
 cabo una toma de decisiones con una serie de entradas, las cuales serán aquellas
 relacionadas con el entorno de un asalto de esgrima, como son las características
 de los tiradores, como se está desarrollando el asalto, etc.

Esta aplicación será desarrollada llevando a cabo una labor de Ingeniería de Conocimiento
 junto con una extracción y análisis de datos para describir el problema, extraer
 el conocimiento de expertos, conceptualizar, formalizar e implementar dicho conocimiento
 de manera entendible para los usuarios de dicha aplicación. Dicha aplicación será un SBC
 el cual aglutine todo el conocimiento de los expertos en su conjunto, además del análisis
 de datos. Dicho sistema tendrá el objetivo darle una respuesta a un tirador de tal forma
 que este tenga un punto de vista más para tomar sus decisiones, de tal modo que le sea mas
 fácil alcanzar la victoria. Además de ayudarle a anteponerse a su rival, también servirá
 como entrenamiento y salir de dudas cuando se quiera mejorar y adquirir conocimiento.

Esto nos lleva a la conclusión de que para alcanzar el objetivo de este TFG tendremos
 que diseñar un sistema de apoyo a la toma de decisión, con acceso mediante una aplicación
 web para darle accesibilidad al programa desde cualquier sitio.

El alcance de este proyecto está basado en los recursos disponibles para realizarlo, tanto personas como tiempo.
 Varios autores han escrito libros para plasmar su conocimiento sobre este deporte,
 ya sea como plantear la gestión de un club, como preparar a los tiradores para competiciones,
 como iniciarlos, etc. Este último caso es el de Elain Cheris hablando sobre los fundamentos
 básicos de la esgrima en las modalidades de florete y espada ya que ambas comparten
 las bases. Este libro Manual de esgrima consta de 160 páginas en el que se habla sobre
 el primer año de aprendizaje de una persona que se inicia en el deporte. Para adquirir
 este conocimiento se requiere de muchas horas de trabajo y entrevistas con profesionales
 por lo que automáticamente descartamos la modalidad de sable, ya que hay poco conocimiento
 reutilizable.

Debido a los motivos expuestos anteriormente la modalidad de sable se dejará para un futuro
 a modo de ampliación. Respecto a la modalidad de florete es cierto que comparten
 las bases pero las técnicas específicas y el conocimiento es totalmente distinto,
 ya que las propias reglas difieren entre ambas modalidades en algunos aspectos,
 por lo que se podrían utilizar partes del desarrollo pero toda la adquisición de
 conocimiento, desarrollo del sistema experto habría que realizarlo partiendo de cero.
 Para ambas modalidades hay que sumar que conseguir expertos resulta de gran dificultad
 actualmente, cosa que en un futuro lejano, tres años, espero solventar. Todo esto ha
 llevado a los objetivos expuestos anteriormente.

\section{Estructura del trabajo}

Pendiente de escribir
