\chapter{Anexo III}
\label{cap:Entrevsitas}

\textbf{Fecha:}


\textbf{Hora:}

\textbf{Lugar:}

\textbf{Asistentes:}

\textbf{Conocimiento anterior a la entrevista:} Síntesis del conocimiento obtenido
 de las entrevistas anteriores.

\textbf{Objetivos de la entrevista:} El objetivo que se pretende alcanzar con la entrevista.

\textbf{Fuentes de conocimiento:} Personas de las cuales se obtiene el conocimiento o
 lo que es lo mismo, las personas que van a ser entrevsitadas.

\textbf{Modo:} Entrevsita estructurada o parcialmente estructurada. Para identificación de
 dichos errores.

\textbf{Planteamiento de la sesión:} En este apartado se muestran las preguntas que se desean
 realizar para obtener el conocimiento.

\textbf{Resultado de la sesión:} Aquí se transcriben las respuestas obtenidas a las preguntas
 del planteamiento de la sesión.

\textbf{Plan de análisis:} Pasos a realizar para analizar.

\textbf{Resultado del análisis:} Resultado final de la entrevista.


A continuación se muestran algunas entrevistas realizadas a partir del formato anterior.

\section{Entrevista 1}

\textbf{Fecha:} Jueves 20 de Abril de 2017.

\textbf{Hora:} 19:40

\textbf{Lugar:} Sala de armas Espada de Calatrava

\textbf{Asistentes:}
  \begin{itemize}
    \item Juan Lomas Rayego (Experto).
    \item Gregorio B. Patiño Esteo (IC).
  \end{itemize}

\textbf{Situación del análisis respecto al modelo general:} Esta entrevista es la primera a realizar dentro del conjunto de entrevistas y diferentes técnicas
 para la adquisición del conocimiento necesario para realizar el prototipo de sistema experto
 (S.E). Esta será la primera entrevista por la que haremos preguntas muy generales y sobre la
 marcha iremos haciendo preguntas sobre las que tengamos dudas.

\textbf{Conocimiento anterior a la entrevsita:} Puesto que es la primera de las entrevistas el conocimiento anterior es nulo, por tanto no
tendremos conocimiento previo exceptuando el adquirido por la investigación previa, que es el
sistema de puntaje y las normas.

\textbf{Objetivos de la entrevista:}
  \begin{enumerate}[(A)]
    \item Identificar Las características en las que hay que fijarse
     en ambos tiradores para determinar la táctica.
    \item Determinar las relaciones en estas características para saber la táctica a seguir.
  \end{enumerate}

\textbf{Fuentes de conocimiento:} Personas de las cuales se obtiene el conocimiento o
 lo que es lo mismo, las personas que van a ser entrevsitadas.

\textbf{Modo:} Entrevsita estructurada o parcialmente estructurada. Para identificación de
 dichos errores.

\textbf{Planteamiento de la sesión:} En este apartado se muestran las preguntas que se desean
 realizar para obtener el conocimiento.

\textbf{Resultado de la sesión:} Aquí se transcriben las respuestas obtenidas a las preguntas
 del planteamiento de la sesión.

\textbf{Plan de análisis:} Pasos a realizar para analizar.

\textbf{Resultado del análisis:} Resultado final de la entrevista.


