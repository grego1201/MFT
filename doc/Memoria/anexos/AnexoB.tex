\chapter{Entrevsitas}
\label{cap:Entrevsitas}

\textbf{Fecha:}


\textbf{Hora:}

\textbf{Lugar:}

\textbf{Asistentes:}

\textbf{Conocimiento anterior a la entrevista:} Síntesis del conocimiento obtenido
 de las entrevistas anteriores.

\textbf{Objetivos de la entrevista:} El objetivo que se pretende alcanzar con la entrevista.

\textbf{Fuentes de conocimiento:} Personas de las cuales se obtiene el conocimiento o
 lo que es lo mismo, las personas que van a ser entrevsitadas.

\textbf{Modo:} Entrevsita estructurada o parcialmente estructurada. Para identificación de
 dichos errores.

\textbf{Planteamiento de la sesión:} En este apartado se muestran las preguntas que se desean
 realizar para obtener el conocimiento.

\textbf{Resultado de la sesión:} Aquí se transcriben las respuestas obtenidas a las preguntas
 del planteamiento de la sesión.

\textbf{Plan de análisis:} Pasos a realizar para analizar.

\textbf{Resultado del análisis:} Resultado final de la entrevista.


A continuación se muestran algunas entrevistas realizadas a partir del formato anterior.

\section{Entrevista 1}

\textbf{Fecha:} Jueves 20 de Abril de 2017.

\textbf{Hora:} 19:40

\textbf{Lugar:} Sala de armas Espada de Calatrava

\textbf{Asistentes:}
  \begin{itemize}
    \item Juan Lomas Rayego (Experto).
    \item Gregorio B. Patiño Esteo (IC).
  \end{itemize}

\textbf{Situación del análisis respecto al modelo general:} Esta entrevista es la primera a realizar dentro del conjunto de entrevistas y diferentes técnicas
 para la adquisición del conocimiento necesario para realizar el prototipo de sistema experto
 (S.E). Esta será la primera entrevista por la que haremos preguntas muy generales y sobre la
 marcha iremos haciendo preguntas sobre las que tengamos dudas.

\textbf{Conocimiento anterior a la entrevsita:} Puesto que es la primera de las entrevistas el conocimiento anterior es nulo, por tanto no
tendremos conocimiento previo exceptuando el adquirido por la investigación previa, que es el
sistema de puntaje y las normas.

\textbf{Objetivos de la entrevista:}
  \begin{enumerate}[(A)]
    \item Identificar Las características en las que hay que fijarse
     en ambos tiradores para determinar la táctica.
    \item Determinar las relaciones en estas características para saber la táctica a seguir.
  \end{enumerate}

\textbf{Fuentes de conocimiento:} Experto.
La razón por la que se llevó a cabo esta elección es que además de ser entrenador es tirador y
tiene bastantes años a sus espaldas que le respaldan, además de haber participado en el circuito
nacional y haber asistido a varias concentraciones de la selección española como técnico. Por
tanto ambos objetivos podrían cumplirse ya que con su experiencia como tirador podría
identificar las características en las que habría que fijarse (objetivo A) y su conocimiento teórico
adquirido como entrenador (con sus respectivos cursos y exámenes) podría relacionarlos para
saber qué hacer en cada caso (objetivo B).

\textbf{Modo:} El modo en el que se hará la entrevista será abierta. Realizando preguntas bastante abiertas
para que el experto pueda expresarse libremente y de esta manera adquirir el mayor glosario
posible y además poder conocer cuáles pueden ser los casos. En caso de que el conocimiento
sea adquirido de una buena forma podría empezarse a preguntar cosas más específicas sobre
ciertas situaciones.

\textbf{Planteamiento de la sesión:} En este apartado se muestran las preguntas que se desean
 realizar para obtener el conocimiento.

\begin{description}
  \item [A1.] ¿Qué es lo primero en lo que te fijas a la hora de planificar una táctica?
  \item [A2.] ¿Cómo determinas si tiene más distancia que tú?
  \item [A3.] ¿Cómo actúa la experiencia a la hora de determinar la táctica?
  \item [A4.] ¿Qué puedes decirme sobre el físico?
\end{description}

\begin{description}
  \item [B1.] ¿Si tiene más distancia que tu como lo valoras?
  \item [B2.] ¿Qué puedes decirme sobre los puños franceses y anatómicos, ventajas y cómo actuar?
  \item [B3.] ¿Echar de la pista o ceder terreno?j
  \item [B4.] ¿Ser agresivo o pasivo?
\end{description}


\textbf{Resultado de la sesión:} Aquí se transcriben las respuestas obtenidas a las preguntas
 del planteamiento de la sesión.

A1. ¿Qué es lo primero en lo que te fijas a la hora de planificar una táctica?

Si tiene más distancia, experiencia y físico.
\\

B1. ¿Si tiene más distancia que tu como lo valoras?

Lo que hago es intentar usar técnicas que me permitan entra una técnica en una distancia en la
que yo puedo tocar y el no, entonces lo que necesito es entrar con hierro o con segunda
intención.
\\

A2. ¿Cómo determinas si tiene más distancia que tú?

Por la altura suele ser un buen indicador otro es si lleva francesa y yo anatómica y luego haces
una valoración.

A3 ¿Cómo actúa la experiencia a la hora de determinar la táctica?

Si es más joven que yo es una persona que no sabe mucho y se siente intimidado por mí por lo
que intento ganarle más rápidamente, o si esta echado hacia delante lo espero, provoco que
cometa errores y le gano en contra.
\\

A4 ¿Qué puedes decirme sobre el físico?
Si tiene más físico que yo, por ejemplo no puedo pretender ser más rápido que una persona que
es más rápida que yo, entones la técnica cambia, obviamente si él es más lento que yo quizás
siendo más ofensivo es buena opción, si es más rápido que yo tengo que esperarme más o jugar
con una segunda intención.
\\

B2 ¿Qué puedes decirme sobre los puños franceses y anatómicos, ventajas y cómo actuar?
Puño francés sus ventajas son alcance y poco más, actuaria con esgrima de poco contacto en
hierro evitando el contacto con el rival y una distancia larga media siempre, favorece un mejor
físico
Anatómica pierdes distancia pero ganas puntería y tienes que entrar en una distancia media baja
y necesita más técnica.
\\

B3 ¿Echar de la pista o ceder terreno?
Si veo que la otra persona no ataca, le llevo a final de pista y no puede moverse o le toco o se
sale. Si me vuelvo loco y es una persona que para muy bien o que al contrataque va muy bien
seguramente me meta muchos puntos, sin embargo, si tiene una esgrima de mucho más ímpetu
dejo que ataque, le dejo corto y le doy después.
\\

B4 ¿Ser agresivo o pasivo?
Normalmente suelo ser más precavido por mi personalidad, pero hay veces que si eres agresivo
puedes intimidar al otro como cuando tienen menos experiencia que tu o es más joven que tú,
a esa gente hay que ganarle rápido antes de que pueda pensar nada. Que es alguien con más
ímpetu que tú, dejas que se precipite y le ganas entonces eres más pasivo
\\

\textbf{Plan de análisis:} Pasos a realizar para analizar.
\begin{itemize}
  \item Identificar términos usados en la entrevista
  \item Identificar características tirador
  \item Comprender ipmortancia y relación de las características de un tirador
  \item Aumentar glosario
\end{itemize}

\textbf{Resultado del análisis:} Resultado final de la entrevista.
Términos:
\begin{itemize}
  \item Distancia
  \item Experiencia
  \item Físico
  \item Altura
  \item Puño
  \item Francés
  \item Anatómico
  \item Técnica
  \item Agresividad
  \item Parada
\end{itemize}

Características de un tirador:
\begin{itemize}
  \item Altura
  \item Físico
  \item Experiencia
  \item Juventud
  \item Agresivo
\end{itemize}

Relación características tirador: \\
Después de la entrevista se ha comprendido que el estado físico de una persona cobra importancia
 en la esgrima. Siendo este un factor importante pero no determinante, el cual se puede
 compensar de otras maneras, ya sea con experiencia u otras técnicas. Quizás la característica
 que destaca mas sobre el resto sería la experiencia, puesto que esta es la que te permitirá
 adaptarte dentro del asalto. De todas formas esto se afianzará en próximas entrevistas.
