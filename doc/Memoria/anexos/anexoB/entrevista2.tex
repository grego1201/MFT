\section{Entrevista 2}

\textbf{Fecha:} Viernes 4 de Mayo de 2017.

\textbf{Hora:} 19:40

\textbf{Lugar:} Sala de armas Espada de Calatrava

\textbf{Asistentes:}
  \begin{itemize}
    \item Juan Lomas Rayego (Experto).
    \item Gregorio B. Patiño Esteo (IC).
  \end{itemize}

\textbf{Situación del análisis respecto al modelo general:} Con la anterior entrevista finalizada
y resueltas todas las dudas mediante medios de comunicación digitales para favorecer la fluidez
de estos mismos, estamos mas cerca de alcanzar el modelo general. Es por esto por lo que esta segunda
entrevista será dedicada a ampliar los conocimientos adquiridos consiguiendo entrar en movimientos
no tan básicos y combinaciones más complejas

\textbf{Conocimiento anterior a la entrevsita:} Partimos de una base de conocimiento adquirido
por la anterior entrevista, en la cual se vieron características básicas. Este junto al anterior
obtenido mediante investigación propia ya nos da una licencia para entender algo sobre tácticas
en esgrima. También podremos formar nuestra propia táctica, pero esta sera muy básica. El conocimiento
que tenemos son los movimientos básicos junto a las características principales.

\textbf{Objetivos de la entrevista:}
  \begin{enumerate}[(A)]
    \item Ampliar base movimientos.
    \item Ampliar base combinaciones.
    \item Ampliar base características.
  \end{enumerate}

\textbf{Fuentes de conocimiento:} Experto.
La razón por la que se llevó a cabo esta elección es que además de ser entrenador es tirador y
tiene bastantes años a sus espaldas que le respaldan, además de haber participado en el circuito
nacional y haber asistido a varias concentraciones de la selección española como técnico. Por
tanto todos los objetivos podrían cumplirse ya que con su experiencia como tirador podría
identificar las características en las que habría que fijarse (objetivo C) y su conocimiento teórico
adquirido como entrenador (con sus respectivos cursos y exámenes) podría relacionarlos para
saber qué hacer en cada caso (objetivo A y B).

\textbf{Modo:} El modo en el que se hará esta entrevista será semi-abierta. Algunas de las preguntas
serán con respuesta totalmente abiertas, mientras que otras de ellas serán propuestas que haremos
nosotros para saber si vamos cogiendo el conocimiento de una manera correcta, y en caso de que no lo fuera
corregirlo.

\textbf{Planteamiento de la sesión:} En este apartado se muestran las preguntas que se desean
 realizar para obtener el conocimiento.

\begin{description}
  \item [A1.] ¿Hay alguna manera de saltarse la guardia del rival?
    %coupé
  \item [A2.] ¿Hay alguna manera de forzar al rival para que ataque?
    %fintas de defensa
  \item [A3.] ¿Hay alguna manera de distraer al rival legalmente?
    %llamada (pie)
\end{description}

\begin{description}
  \item [B1.] ¿Hay alguna manera de engañar al rival con segundas intenciones?
    %finta y pase
  \item [B2.] ¿Hay alguna manera de engañar al rival en la distancia?
    %juego de piernas marcha-fondo
\end{description}

\begin{description}
  \item [C1.] ¿Como de importante es que el tirador sepa hacer bien las paradas?
  \item [C2.] ¿Como de importante es que el tirador sepa atacar bien?
  \item [C3.] ¿Como de importante es que el tirador caiga en las trampas y engaños?
  \item [C4.] ¿Como influye el paso del tiempo en el asalto?
  \item [C5.] ¿Como actua la confianza en el tirador dentro del asalto?
  \item [C6.] ¿Como influye la confianza en el tirador a la hora de planificar la táctica?
\end{description}

\textbf{Resultado de la sesión:} Aquí se transcriben las respuestas obtenidas a las preguntas
 del planteamiento de la sesión.


A1.¿Hay alguna manera de saltarse la guardia del rival?

Sí, existe un movimiento llamado lanzado o mas conocido como coupé el cual te permite saltar
la guardia del rival. La única defensa ante este ataque es quitar el brazo o intentar tocar
con un contra-ataque al rival para que tu luz se encienda antes que la del otro, asumiendo los
riesgos de fallar o no tocar con suficiente antelación.
\\


A2. ¿Hay alguna manera de forzar al rival para que ataque?
Sí. Para ello están las fintas. Las fintas es un movimiento por el cual tu engañas a tu rival
para provocar la acción que tu quieras. Por ejemplo, si yo quiero hacer algún movimiento, pero para
este necesito que intentes tocarme en el pie. En este caso esperar a que el rival vaya al pie
puede ser muy largo o no darse el caso. Por esto mismo podrémos provocar al rival dejando el pie
al descubierto para que este tenga mas ganas de ir.
\\


A3. ¿Hay alguna manera de distraer al rival legalmente?
Sí. Toda acción que hagas es legal siempre y cuando no intimides al rival, no pongas en riesgo tu
seguridad ni sea antideportivo. Por ejemplo, ponerte a gritar podría ser algo antideportivo además
de intimidar al rival, por tanto eso no lo podrías hacer. Algo que si podrías hacer sería dar pisotones
de vez en cuando en el suelo. Esto hará que tu rival se distraiga y entonces tu aprovechar para lanzar
un ataque. Hay gente que esto no lo considera del todo honorable pero eso ya a critero de cada uno.
Algo con mas de estilo es dar golpes con tu espada en la suya, suaves y fuertes, variando entre ellos.
De esta manera tu rival estará mas pendiente de tu espada y podrás aprovechar para disumular otras
acciones con el cuerpo, como ganarle distancia.
\\


B1. ¿Hay alguna manera de engañar al rival con segundas intenciones?

Sí. Al igual que hablabamos antes que puedes hacer acciones con el cuerpo para distraer al rival,
también las puedes hacer con la propia espada a modo de acciones. Puedes hacer un movimiento
amagando que vas a atacar a un punto suyo en concreto y cuando el vaya a defenderse, entonces
cambiar el objetivo y atacar a otro sitio. Esto es conocido como 1-2 o mas técnicamente finta-pase.
\\


B2. ¿Hay alguna manera de engañar al rival en la distancia?

Sí. Con la propia guardia puedes engañarle, si la avanzas el rival se pensará que estás más cerca
de lo que realmente te encuentras ya que tu punta se encontrará mas cerca de él. Por otro lado
si la retrasas conseguirás que piense que estás mas lejos, pudiendo aprovechar esto para tocar
con mayor facilidad en tocados a corta distancia. Por otro lado también se puede hacer con un
movimiento constante de piernas en el que la distancia de este no sea siempre la misma.
\\


C1. ¿Como de importante es que el tirador sepa hacer bien las paradas?

Es tan importante como que esto determinará si podemos ser mas agresivos o no en nuestros ataques.
Incluso siendo algo extremistas, podría ser lo que indique si debemos atacar durante el asalto.
Ante un oponente que ejecute a la perfección sus paradas y no tenga huecos en la defensa, lo mejor
no será atacar puesto que esto provocará en la mayoría de los casos que cometamos errores atacando
y de ese modo obtendremos puntos en contra, y eso nunca lo queremos. Por el contrario, si el rival
tiene muchos huecos en defensa nos centraremos en atacar.
\\

C2. ¿Como de importante es que el tirador sepa atacar bien?

Es lo mismo que hemos hablado antes solo que se invertirían los roles entre tiradores. No puedes
intentar defender si el rival es muy superior a ti en su ataque.
\\

C3. ¿Como de importante es que el tirador caiga en las trampas y engaños?

Otro factor que dirá si debemos hacer ataques indirectos o directos. Si el rival siempre cae en
nuestros engaños lo mejor será elaborar mas el tocado para asegurarlo. Por el contrario si el
rival nunca cae en nuestros engaños lo mejor será centrarnos en la explosividad e ir lo mas recto
posible.
\\

C4. ¿Como influye el paso del tiempo en el asalto?

Esto influirá en como de agresivos han de ser los tiradores en función del resultado. En caso de
que vayas perdiendo y el resultado sea muy distante, el tirador que pierda deberá intentar dejar
pasar el menor tiempo posible, puesto que este juega en su contra. Por otro lado, el que gane
podrá permitirse la licencia de dejar correr el cronometro. Esto es debido a que cuanto mas tiempo pase
estando el por encima, ma
\\

C5. ¿Como actua la confianza en el tirador dentro del asalto?

Esto es bastante dificil de determinar. Cuando un tirador esté repleto de confianza intentará
tomar decisiones mas arriesgadas, mientras que cuando esté bajo de esta lo mas probable es que
intente actuar bajo seguro esperando los errores del rival. Por tanto este podría ser un
indicativo sobre la confianza que tiene el rival en él. Pero de nuevo, insisto en que cada
persona actua de una forma distinta ante la ausencia o exceso de confianza.
\\

C6. ¿Como influye la confianza en el tirador a la hora de planificar la táctica?

Bien esto es algo complejo. Habrá que distinguir por dos partes. En cuanto a nuestro tirador
hay que tener cuidado con el exceso de confianza puesto que esto nos hará cometer errores.
Por otro lado hay que tener cuidado también con la falta de confianza, puesto que esto hará
que dudemos y perderemos tiempo en las acciones, lo que provocará que sea mas fácil que nos toquen.
\\

\textbf{Plan de análisis:} Pasos a realizar para analizar.
\begin{itemize}
  \item Identificar términos usados en la entrevista
  \item Identificar características nuevas tirador
  \item Comprender importancia y relación de las nuevas características de un tirador
  \item Aumentar glosario
\end{itemize}


\textbf{Resultado del análisis:} Resultado final de la entrevista.
Términos:
\begin{itemize}
  \item Coupé
  \item Guardia
  \item Defensa
  \item Contra-ataque
  \item Finta
  \item Provocar
  \item Engañar
  \item Distraer
  \item Llamada
  \item Finta-Pase
\end{itemize}

Características de un tirador:
\begin{itemize}
  \item Reflejos
  \item Capacidad defensiva
  \item Capacidad ofensiva
  \item Confianza
  \item Explosividad
\end{itemize}

Relación nuevas características tirador: \\
Se habló sobre los reflejos y la explosividad, por lo que se entiende que la velocidad
en un tirador es importante. Esto hace que haya que tener realmente en cuenta.
Por otro lado también se habló de las capacidades del tirador, tanto ofensivas como defensivas.
Esto será importante a la hora de decantarnos por una táctica u otra.
También se mencionó la confianza que tiene un tirador en si mismo. Esto se puede aprovechar
tanto en el rival como en nosotros. Detectando si tiene exceso o escasez de esta podremos ser
mas agresivos o tendremos que ser mas conservadores.
