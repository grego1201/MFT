% Ajustes definidos por el usuario

% Control de espaciado entre párrafos e indentación
\setlength{\parskip}{1.3ex plus 0.2ex minus 0.2ex} % Espacio entre párrafos (más versatil cuando se fija como una rubber length (longitud elástica)
%\setlength{\parskip}{2mm plus 0.2mm minus 0.2mm} (Clase esi-tfg.cls por D. Villa)
%\setlength{\parindent}{8ex} % Valor por defecto de la indentación en español. 
%Sólo es preciso activar para modificar el valor por defecto (indicar valor cero para anular)


% Ajuste de cabecera y pie de página (paquete fancyhdr)
\fancyhf{} % Reset de la cabecera y pie
% En las páginas impares, parte izquierda del encabezado, aparecerá el nombre de capítulo
\fancyhead[LO]{\sffamily\leftmark} 
% En las páginas pares, parte derecha del encabezado, aparecerá el nombre de sección
\fancyhead[RE]{\sffamily\rightmark} 
% Números de página en las esquinas de los encabezados
\fancyhead[RO,LE]{\sffamily\thepage}

% Formato para el capítulo: N. Nombre
\renewcommand{\chaptermark}[1]{\markboth{\textbf{\thechapter.#1}}{}}
% Formato para la sección: N.M. Nombre
\renewcommand{\sectionmark}[1]{\markright{\textbf{\thesection. #1}}} 

% Ancho de la línea horizontal bajo el encabezado
\renewcommand{\headrulewidth}{0.6pt} 
% Ancho de la línea horizontal sobre el pie (en este ejemplo está vacío)
%\renewcommand{\footrulewidth}{0.6pt} 

%\setlength{\headheight}{16pt} % Por defecto \headheight 12pt, pero se agranda al emplear fancyhdr
\setlength{\headheight}{1.5\headheight} % Aumenta la altura del encabezado en una vez y media


% Evita que la última página de cap. tenga cabecera y pie 
% si dicha página está en blanco (para clase book)
\makeatletter
\def\cleardoublepage{\clearpage\if@twoside
\ifodd\c@page
\else\hbox{}\thispagestyle{empty}\newpage
\if@twocolumn\hbox{}\newpage\fi\fi\fi}
\makeatother



%====================================
% Configuración inicio capítulos (paquete titlesec)
\newcommand{\bigrule}{\titlerule[0.5mm]}

\titleformat{\chapter}[display] % cambiamos el formato de los capítulos
{\bfseries\Huge\sffamily} % por defecto se usarán caracteres de tamaño \Huge en negrita y familia sans serif
{% contenido de la etiqueta
\titlerule % línea horizontal
\filleft % texto alineado a la derecha
\Large{\MakeUppercase\chaptertitlename}\ % "Capítulo" o "Apéndice" en tamaño \Large en lugar de \Huge
\Large\thechapter} % número de capítulo en tamaño \Large
{0mm} % espacio mínimo entre etiqueta y cuerpo
{\filleft\MakeUppercase} % texto del cuerpo alineado a la derecha
[\vspace{0.5mm} \bigrule] % después del cuerpo, dejar espacio vertical y trazar línea horizontal gruesa

%====================================
% Tipo de letra empleado en títulos de secciones (paquete sectsty)
\sectionfont{\sffamily\bfseries\MakeUppercase}
\subsectionfont{\sffamily\bfseries}
\subsubsectionfont{\sffamily\bfseries}

%====================================
% Alguna definiciones de nivel de gris
\definecolor{gris97}{gray}{.97}
\definecolor{gris95}{gray}{.95}
\definecolor{gris75}{gray}{.75}
\definecolor{gris45}{gray}{.45}


% Personalización del entorno lstlisting (ver documentación paquete listings para más infor.)
   
\lstset{ % Estilo por defecto
	basicstyle={\footnotesize\ttfamily}, % Estilo básico para el texto
	%stringstyle=\textsl,        % Estilo para las cadenas
	stringstyle={\color{Red1}\ttfamily\bfseries},
	commentstyle={\color{Green4}\sffamily\bfseries},% Estilo para los comentarios
	keywordstyle={\color{Blue1}\bfseries},% Estilo para las palabras clave
	%	keywordstyle=[1]\textbf,    % Posibilidad de particularizar el estilo 
	%	keywordstyle=[2]\textbf,    %
	%	keywordstyle=[3]\textbf,    %
	%	keywordstyle=[4]\textbf,    %
	%	deletekeywords={}, 			% Quita keywords separadas por comas
	captionpos=t,               % Ajusta la posición de títulos 
	numbers=left,               % Posición de números de línea
	numberstyle={\tiny\sffamily\bfseries},          % Tamaño del número de línea
	numberfirstline=false,
	firstnumber=1, 				%  Nº de la primera línea
	stepnumber=1,               % Paso de línea numerada
	numbersep=10pt,             % Separación del texto al número de línea
	tabsize=2,                  % Tamaño del tabulador
	extendedchars=\true,        % Gestiona en empleo de caracteres extendidos utf8
	texcl=true,				    % Necesario para unicode en los comentarios
	breaklines=true,            % Ajusta división automática de líneas
	breakatwhitespace=true,     %
	frame=single,               % none, leftline, topline, bottomline, lines, single, shadowbox 
	frameround=tttt, 			% Redondea los bordes del frame
	rulecolor={\color{IndianRed}},    % Color del frame
	showspaces=false,           % Muestra espacios en blanco
	showtabs=false,             % Muestra tabuladores
	showstringspaces=true,      % Muestra espacios en blanco en las cadenas        
	xleftmargin=1cm,xrightmargin=1cm,
	breaklines=true,
	%	framexleftmargin=17pt
	%	framexrightmargin=5pt,
	%	framexbottommargin=4pt,
	backgroundcolor={\color{Azure1}} % Color del fondo
}


%Estilo definido para lenguaje C
\lstdefinestyle{C}{%
	language=C,
	frame=L,
	rulesep=.1pt,
	rulecolor=\color{black},
}


% Estilo definido para comandos de consola
\lstdefinestyle{consola}{%
	basicstyle={\color{White}\scriptsize\bf\ttfamily},
	backgroundcolor={\color{Black}},
	frame=none,
	showspaces=true
}




% Comandos definidos por el usuario
%====================================
% Comandos que definen datos del documento
\makeatletter
\newcommand{\tituloPrimera}[1]{\newcommand{\@tituloPrimera}{#1}}
\newcommand{\tituloSegunda}[1]{\newcommand{\@tituloSegunda}{#1}}
\newcommand{\titulo}[1]{\newcommand{\@titulo}{#1}\renewcommand{\@title}{#1}}
\newcommand{\tipoDoc}[1]{\newcommand{\@tipoDoc}{#1}}
\newcommand{\autor}[1]{\newcommand{\@autor}{#1}\renewcommand{\@author}{#1}}
\newcommand{\email}[1]{\newcommand{\@email}{\url{#1}}}
\newcommand{\director}[1]{\newcommand{\@director}{#1}}
\newcommand{\codirector}[1]{\newcommand{\@codirector}{#1}}
\newcommand{\tutor}[1]{\newcommand{\@tutor}{#1}}
\newcommand{\instEdu}[1]{\newcommand{\@instEdu}{#1}}
\newcommand{\centroEdu}[1]{\newcommand{\@centroEdu}{#1}}
\newcommand{\deptoEduPrimera}[1]{\newcommand{\@deptoEduPrimera}{#1}}
\newcommand{\deptoEduSegunda}[1]{\newcommand{\@deptoEduSegunda}{#1}}
\newcommand{\escudo}[1]{\newcommand{\@escudo}{#1}}
\newcommand{\titulacion}[1]{\newcommand{\@titulacion}{#1}}
\newcommand{\especialidad}[1]{\newcommand{\@especialidad}{#1}}
\newcommand{\fechaDef}[1]{\newcommand{\@fechaDef}{#1}}
\newcommand{\mesDef}[1]{\newcommand{\@mesDef}{#1}}
\newcommand{\yearDef}[1]{\newcommand{\@yearDef}{#1}}
\newcommand{\lugarDef}[1]{\newcommand{\@lugarDef}{#1}}
\makeatother


%====================================
% Portada (tfg)
% Portada (aquí gralmente no habrá que editar nada)
\makeatletter
\newcommand{\portadaTFG}{%
	\begin{titlepage}
		\begin{center}
			\includegraphics[width=3.5cm]{\@escudo}\vspace{1cm}
			
			{\LARGE \textbf{\@instEdu\\[1.5\parskip]
					\@centroEdu\\[2cm]
					\@titulacion}}\\[0.5cm]
			{\large \textbf{\@especialidad}}\\[1.5cm]
			{\LARGE \textbf{\@tipoDoc}}\\[1cm]
			
			{\LARGE \@tituloPrimera}\\ \smallskip%			
			\ifdefined\@tituloSegunda{\LARGE	\@tituloSegunda}\\[3cm]
			\else \vspace{3cm}
			\fi
			
			{\Large \@autor}\vfill%
		\end{center}
		
		\begin{flushright}
			{\Large \@fechaDef}
		\end{flushright}
		
		\cleardoublepage
\end{titlepage}}
\makeatother

%====================================
% Página inicial (tfg)
% Página inicial (es como la portada, añade Director/es o Tutor)
\makeatletter
\newcommand{\portadillaTFG}{%
	\begin{center}
		\includegraphics[width=3.5cm]{\@escudo}\\[1.5cm]
		
		{\LARGE \textbf{\@instEdu \\[1.5\parskip]
				\@centroEdu}}\\[0.5cm]
		{\Large \textbf{\@deptoEduPrimera}}\\ \smallskip%
		\ifdefined\@deptoEduSegunda{\Large \textbf{\@deptoEduSegunda}}\\[0.5cm]
		\else \vspace{0.5cm}
		\fi
		{\large \textbf{\@especialidad}}\\[1.5cm]
		
		{\LARGE \textbf{\@tipoDoc}}\\[1cm]
		
		
		{\LARGE \textbf{\@tituloPrimera}}\\ \smallskip%		
		\ifdefined\@tituloSegunda{\LARGE \textbf{\@tituloSegunda}}\\[3cm]
		\else \vspace{3cm}
		\fi
	\end{center}
	
	\begin{flushleft}
		{\Large Autor(a): \@autor} \\ \bigskip%
		{\Large Director(a): \@director} \\ \bigskip%
		% Si hay definido un codirector se añade automáticamente la línea siguiente
		\ifdefined\@codirector {\Large Director(a): \@codirector} \fi 
	\end{flushleft}
	\vfill%
	
	\begin{flushright}
		{\Large \@fechaDef}
	\end{flushright}
	
	\newpage}
\makeatother

%====================================
% Página tribunal (tfg)
% Tribunal
\makeatletter
\newcommand{\tribunalTFG}{
	{\flushright \LARGE \textsc{Tribunal:}}
	
	\vspace*{\stretch{0.5}}
	\hspace*{1cm}{\Large Presidente: \hrulefill}
	
	\vspace*{\stretch{0.5}}
	\hspace*{1cm}{\Large Vocal: \hrulefill}
	
	\vspace*{\stretch{0.5}}
	\hspace*{1cm}{\Large Secretario: \hrulefill}
	
	\vspace*{\stretch{0.5}}
	{\flushright \LARGE \textsc{Fecha de defensa:} \hrulefill}
	
	\vspace*{\stretch{1.5}}
	{\flushright \LARGE \textsc{Calificación:} \hrulefill}
	
	\vspace*{\stretch{2.5}}
	\begin{center}
		\begin{tabularx}{\linewidth}{X X X}
			{\large \textsc{Presidente}} & {\large \textsc{Vocal}} & {\large \textsc{Secretario}}\\[2.5cm]
			Fdo.: & Fdo.: & Fdo.:		
		\end{tabularx}
	\end{center}
	\cleardoublepage}
\makeatother



%====================================
% Comando para incluir la dedicatoria
% Con la opción stretch se puede colocar verticalmente la dedicatoria de forma relativa para que quede el doble de espacio por debajo que por encima
\newcommand{\dedicado}[1]{ % Dedicatoria
	\null\vspace{\stretch{1}}
	\begin{flushright}
	\emph{#1}
	\end{flushright}
	\vspace{\stretch{2}}\null
	\cleardoublepage
}



%====================================
% Créditos y licencia (opcional, 1 pág.)
% Todas las obras en gral. deberían presentar información relativa a la propiedad intelectual del contenido y condiciones bajo las cuales se puede distribuir y reproducir
\makeatletter
\newcommand{\creditos}[2]{%
	\null\vspace{6cm}
	{\small \noindent \@titulo\\
	\textcopyright{} \@autor, \@yearDef\\[1cm]
	#1}\\
	\includegraphics[width=0.15\linewidth]{#2}
	
	\clearpage
}
\makeatother




%====================================
% Añade el comando \tecla para crear indicaciones de pulsación de teclas
\usetikzlibrary{shadows} % Necesario para poder crear nuevo comando de indicación de pulsación de tecla.
\newcommand*\tecla[1]{%   
  \tikz[baseline=(key.base)]
    \node[%
      draw,
      fill=white,
      drop shadow={%
	      shadow xshift=0.25ex,
	      shadow yshift=-0.25ex,
	      fill=black,
	      opacity=0.75
      },
      rectangle,
      rounded corners=2pt,
      inner sep=1pt,
      line width=0.5pt,
      font=\scriptsize\sffamily
    ](key) {#1\strut}
  ;
}

%====================================
% Desactivación de división de palabras. 
% Uso: \nodivide o \nodivide[<n>]
\newcommand{\nodivide}[1][10000]{%
	\hyphenpenalty=#1 % Valor típico: hasta 10000
	\exhyphenpenalty=#1 % Valor típico: hasta 10000
	\sloppy
}

%====================================
% Desactivación de división de palabras. 
% Uso: \nowidowandorphan o \nowidowandorphan[<n>]
\newcommand{\nowidowandorphan}[1][10000]{%
	\clubpenalty=#1  % % Valor típico: hasta 10000
	\widowpenalty=#1 % % Valor típico: hasta 10000
}

%====================================
% Código para evitar la división de notas al pie en págs. diferentes
% Uso: \nodividenotas o \nodividenotas[<n>]
\newcommand{\nodividenotes}[1][10000]{%
	\interfootnotelinepenalty=#1 % Valor típico: hasta 10000
}


%====================================
% Creación de un contador nuevo para almacenar el nº de páginas actual
% OJO: Debe ir antes de \mainmatter (antes de que se reinicie en cnt page)
\newcommand{\savepagecnt}{%
	\newcounter{totpages}
	\setcounter{totpages}{\value{page}}
	\addtocounter{totpages}{1}
}

%====================================
% Continuación de la paginación desde el valor almacenado en \totpages
\newcommand{\contpagination}{%
	\setcounter{page}{\value{totpages}}
}

%====================================
% Limpia las cabeceras de la primera página de capítulo
\newcommand{\cleanhdfirst}{%
	\fancypagestyle{plain}{%
		\fancyhf{}%
		\renewcommand{\headrulewidth}{0pt}
		\renewcommand{\footrulewidth}{0pt}
	}
}

%====================================
% Añadido de entorno abstract en clase book
% donde no está definido por defecto
\newenvironment{abstract}%
{\cleardoublepage\null \vfill\begin{center}%
\bfseries \sffamily \abstractname \end{center}}%
{\vfill \null}


%====================================
% Comando para incluir una nota de aviso al margen.
% Si el margen no es suficientemente amplio puede generar un Overfull \hbox
\newcommand{\ojo}[1]{%
	\marginpar{\footnotesize\raggedright\ding{42} #1}}
\newcommand{\uju}{%
	\renewcommand{\ojo}[1]{}}



